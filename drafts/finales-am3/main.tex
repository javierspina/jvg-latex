\documentclass[9pt,a4paper]{extarticle}
\usepackage[utf8]{inputenc}
\usepackage[spanish]{babel}
\usepackage[top=0.25in, bottom=0.25in, left=0.4in, right=0.4in]{geometry}
\usepackage{amsmath, amssymb, amsfonts}
\usepackage{graphicx}

%letra en arial
%\usepackage{helvet}
%\renewcommand{\familydefault}{\sfdefault}

\usepackage{enumerate}% http://ctan.org/pkg/enumerate
\newcommand{\plus}{\scalebox{0.6}{$+$}}

\begin{document}
\pagenumbering{gobble}
\hrule
\center{Final de Análisis Matemático III}
%\center {3/10/2017}
\begin{enumerate}


    \item Hallar la solución particular de la siguiente ecuación diferencial $$y' -2y=e^t / y(0)=1$$
    \begin{enumerate}
        \item Resolver por la transformada de Laplace
        \item Resolver por otro método distinto
    \end{enumerate}
    
    \item Calular, si existe, el siguiente límite. $$\lim_{z \to 0} \dfrac{\Bar{z}+iz^2}{|z|}$$
    
    \item Realizar la transformación y graficar ambas en planos superpuestos:
    \begin{enumerate}
        \item $|z-2i|=2$ mediante $\dfrac{1}{z}$
        \item $(\Re[z]-1)^2 +(\Im[z]+2)^2 = 4$ mediante $w=3z-2i$
    \end{enumerate}
    
    \item Calcular la siguiente integral a partir de los teoremas de integración
    $$\oint_{C^{\plus}} \dfrac{z^4+i}{z-i}^3 dz$$
    
    \item Hallar el desarrollo por serie de Laurent de $\dfrac{1}{z^3 (1-z^2)}$ indicando centro de convergencia y residuo. Luego, calcular la integral para la curva $C: |z|=\dfrac{1}{2}$

    
    \item  (Teórico) Condición necesaria y suficiente para la existencia de la derivada y la expresión de la derivada.
    \item (Teórico) Pruebe que en una transformación inversa las rectas y las circunferencias se transformen en rectas y circunferencias.
\end{enumerate}
\hrule

\center{Final de Análisis Matemático III}
%\center {3/10/2017}
\begin{enumerate}
    \item Etudiar la continuidad de la siguiente función
    $$
     f(z) = \left\{
	       \begin{array}{cc}
		 \dfrac{z^2-|z|}{\Bar{z}}      & \mathrm{si\ } z \neq 0 \\
		 &\\
		 0     & \mathrm{si\ } z = 0
	       \end{array}
	     \right.
$$
\item Hallar la solución particular de la ecuación diferencial $y' -4y=x$ tal que $y(0)=0$, mediante
\begin{enumerate}
    \item Transformada de Laplace.
    \item otro método distinto
\end{enumerate}

\item Hallar la conjugada armónica de $v(x,y)=2xy-x$ tal que $v(x,y)=\Im[f(z)]$ sabiendo que $f(i)=2$. Descubrir su conjunto de ceros.

\item Calcular la siguiente integral
$$
\oint_{C^{\plus}} \dfrac{cos(z-\pi)}{z^2(z-\pi)} dz
$$
\begin{enumerate}
    \item con $C: |z|=1$
    \item con $C: |z-4|=2$
\end{enumerate}
\item Hallar el desarrollo por serie de Laurent de $f(z)=(z-1)e^{\frac{1}{z-1}}$ con $z_0 =1$ y luego calcular la integral
\item \begin{enumerate}
    \item Demostrar que una transformación por inversión transforma rectas y circunferencias, en rectas y circunferencias.
    \item Enunciar los pasos para realizar una transformación bilineal.
\end{enumerate}
\item \begin{enumerate}
    \item Enunciar la expresión de la derivada, a través de las ecuacions de Cauchy-Riemann en coordenadas polares.
    \item Demostrar que la integral de una función analítica a lo largo de una trayectoria abierta es independiente de dicha trayectoria, siempre que $f$ se mantenga analítica entre la curva dada y la elegida.
\end{enumerate}
\end{enumerate}
\hrule

\newpage
\hrule

\center{Final de Análisis Matemático III}
%\center {3/10/2017}
\begin{enumerate}
\item Hallar la solución particular de la ecuación diferencial $y' +y=e^{-2t}$ tal que $y(0)=0$, mediante
\begin{enumerate}
    \item Transformada de Laplace.
    \item otro método distinto
\end{enumerate}

\item Desarrollar $g(z)=\dfrac{4}{z^2(1+z)}$ en $z_0 =0$ e indicar campo de convergencia. Hallar, mediante el teorema de los residuios, $\displaystyle \oint_{C^{\plus}} g(z)dz$ con $C: |z|=\dfrac{1}{8}$

\item Hallar el transformado de $\Re[z-1]+\Im[z+i]=2$ a través de $w=z^{-1}$. Graficar.

\item Sea $v(x,y)=6xy+2y$ tal que $v(x,y)=\Im[f(z)]$, hallar $f'(z)$

\item Hallar la integral $\displaystyle \oint_{C^{\plus}} \dfrac{5dz}{(z+2i)(z-i)}$ con $C: |z|=\dfrac{3}{2}$. ¿Qué pasaría si $C: |z|=\dfrac{1}{2}$?

\item Demostrar que Laplace es un operador lineal

\item Enunciar las ecuaciones de Cauchy-Riemann en coordenadas polares

\item Deducir alguna ecuación diferencial de primer orden.

\end{enumerate}
\hrule

\end{document}