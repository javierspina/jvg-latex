%--------------------------------
%	Paquetes y configuraciones 
%--------------------------------

\documentclass[a4paper,12pt]{article}
\usepackage[utf8x]{inputenc}
\usepackage[spanish]{babel}
\usepackage{amsmath}
\usepackage{graphicx}
%\usepackage{showframe}
\usepackage[margin=1in]{geometry}
\usepackage{empheq}
\usepackage[most]{tcolorbox}
\usepackage{pgfplots}
\pgfplotsset{compat=newest}
\usepgfplotslibrary{patchplots}
\usepackage{tikz}
\usepackage{verbatim}
\usetikzlibrary{arrows,intersections}

\newtcbox{\mymath}[1][]{
    nobeforeafter, math upper, tcbox raise base,
    enhanced, colframe=blue!30!black,
    colback=blue!30,#1}


%\PassOptionsToPackage{svgnames}{xcolor}
%\usepackage{lipsum}
\tcbuselibrary{skins,breakable}
%\usetikzlibrary{shadings,shadows}

%\newenvironment{block}[1]{%
%    \tcolorbox[beamer,%
%    noparskip,breakable,
%    title=#1]}%
%    {\endtcolorbox}

%----------------------------
%   Comienzo del documento
%----------------------------

\begin{document}
    %\sffamily % Fuente sans serif
    \newcommand{\HRule}{\rule{\linewidth}{0.5mm}}
    
    %-----------------
    %   Portada
    %-----------------
    
\begin{titlepage}
\center

%\includegraphics[scale=0.15]{logo.png}\\[1cm] % Descomentar para usar el logo del JVG
\includegraphics[scale=0.4]{pin_matematica.jpg}\\[1cm] % Descomentar para usar el logo de Matemática
\textsc{\Large Instituto Superior del Profesorado}\\[0.3cm]
\textsc{\LARGE Dr. Joaquín V. González}\\[1cm]
\textsc{\large Profesorado de Educación Superior en Matemática}\\[0.5cm]

\HRule \\[0.4cm]
{\huge \bfseries Análisis Matemático III}\\[0.3cm]
\Large Apuntes de clase\\
\HRule \\[1.5cm]

\begin{minipage}{0.4\textwidth}
\begin{flushleft} 
\large
\emph{Alumno:}\\
Javier Spina
\end{flushleft}
\end{minipage}
~
\begin{minipage}{0.4\textwidth}
\begin{flushright} 
\large
\emph{Docente:}\\
Fabián Nouche
\end{flushright}
\end{minipage}\\[3cm]

\vfill
{\large \today}

\restoregeometry
\end{titlepage}

\tableofcontents
\pagebreak

\section{Ecuaciones Diferenciales}
\subsection{Introducción}
\paragraph{Problema:}
Encontrar una curva cuya recta tangente, en cada uno de sus puntos, tenga pendiente igual al doble del cociente entre la ordenada y la abscisa.\\

[inserte gráfico aquí]

Recordamos que:

[inserte gráfico aquí]

\begin{minipage}{0.4\textwidth}
%\begin{flushleft} 
$$\tan \alpha = \frac{\overline{OQ}}{\overline{OP}}$$
$$\tan \alpha =\frac{0-b}{a-0}$$
\begin{equation} \label{eq:1}
\tan \alpha = -\frac{b}{a}
\end{equation}
%\end{flushleft}
\end{minipage}
~
\begin{minipage}{0.4\textwidth}
%\begin{flushright} 
y por otro lado
$$y = mx + b$$
$$0 = ma + b$$
\begin{equation} \label{eq:2}
m = -\frac{b}{a}
\end{equation}
%\end{flushright}
\end{minipage}

de \ref{eq:1} y \ref{eq:2}

$$ m = \tan \alpha $$

Volviendo al problema:

$$y = mx + b$$
$$m=2\frac{y}{x}$$
$$\tan \ alpha = 2\frac{y}{x}$$
\begin{empheq}[box=\tcbhighmath]{equation}
    y' = 2 \frac{y}{x}
\end{empheq}


%----------------------
%   Nota
%----------------------
\tcolorbox[beamer,title=Nota]
\LaTeX{} is great at typesetting mathematics. Let $X_1, X_2, \ldots, X_n$ be a sequence of independent and identically distributed random variables with $\text{E}[X_i] = \mu$ and $\text{Var}[X_i] = \sigma^2 < \infty$, and let
$$S_n = \frac{X_1 + X_2 + \cdots + X_n}{n}
      = \frac{1}{n}\sum_{i}^{n} X_i$$
denote their mean. Then as $n$ approaches infinity, the random variables $\sqrt{n}(S_n - \mu)$ converge in distribution to a normal $\mathcal{N}(0, \sigma^2)$.
\endtcolorbox



\begin{empheq}[box=\tcbhighmath]{equation*}
    c_i = \langle\psi|\phi\rangle
\end{empheq}
\begin{empheq}[box=\tcbhighmath]{equation*}
    S_n = \frac{X_1 + X_2 + \cdots + X_n}{n}
      = \frac{1}{n}\sum_{i}^{n} X_i
\end{empheq}

\tcbset{highlight math style={boxsep=5mm,colback=blue!30!red!30!white}}

\begin{empheq}[box=\tcbhighmath]{equation*}
    c_i = \langle\psi|\phi\rangle
\end{empheq}

\begin{empheq}[box=\mymath]{equation*}
    c_i = \langle\psi|\phi\rangle
\end{empheq}

\begin{empheq}[box={\mymath[colback=red!30,drop lifted shadow, sharp corners]}]{equation*}
    c_i = \langle\psi|\phi\rangle
\end{empheq}


\section{Introduction}

Your introduction goes here! Some examples of commonly used commands and features are listed below, to help you get started.

If you have a question, please use the support box in the bottom right of the screen to get in touch. 

\section{Some \LaTeX{} Examples}
\label{sec:examples}

\subsection{Sections}

Use section and subsection commands to organize your document. \LaTeX{} handles all the formatting and numbering automatically. Use ref and label commands for cross-references.

\subsection{Comments}
\subsubsection{Whale hello there}

\begin{tikzpicture}
    \begin{axis}[xmax=9,ymax=9,samples=50,grid=major,xlabel={x axis},ylabel={y axis},title={Linear and Quadratic Functions}]
    \addplot[blue, ultra thick](x,x*x);
    \addplot[red, ultra thick](x,x/2);
\end{axis}
\end{tikzpicture}

\begin{tikzpicture} 
    \begin{axis}[ xlabel=Cost, ylabel=Error] 
        \addplot[color=red,mark=x] coordinates {
        (2,-2.8559703) 
        (3,-3.5301677) 
        (4,-4.3050655) 
        (5,-5.1413136) 
        (6,-6.0322865) 
        (7,-6.9675052) 
        (8,-7.9377747)
    }; 
    \end{axis} 
\end{tikzpicture}

\begin{tikzpicture}
    \begin{axis}
    \addplot3[patch,patch refines=3,
		shader=faceted interp,
		patch type=biquadratic] 
    table[z expr=x^2-y^2]
    {
        x  y
        -2 -2
        2  -2
        2  2
        -2 2
        0  -2
        2  0
        0  2
        -2 0
        0  0
    };
    \end{axis}
\end{tikzpicture}

\subsection{Tables and Figures}

Use the table and tabular commands for basic tables --- see Table~\ref{tab:widgets}, for example. You can upload a figure (JPEG, PNG or PDF) using the files menu. To include it in your document, use the includegraphics command as in the code for Figure~\ref{fig:frog} below.

% Commands to include a figure:
\begin{figure}
\centering
\caption{\label{fig:frog}This is a figure caption.}
\end{figure}

\begin{table}
\centering
\begin{tabular}{l|r}
Item & Quantity \\\hline
Widgets & 42 \\
Gadgets & 13
\end{tabular}
\caption{\label{tab:widgets}An example table.}
\end{table}

\subsection{Mathematics}

\LaTeX{} is great at typesetting mathematics. Let $X_1, X_2, \ldots, X_n$ be a sequence of independent and identically distributed random variables with $\text{E}[X_i] = \mu$ and $\text{Var}[X_i] = \sigma^2 < \infty$, and let
$$S_n = \frac{X_1 + X_2 + \cdots + X_n}{n}
      = \frac{1}{n}\sum_{i}^{n} X_i$$
denote their mean. Then as $n$ approaches infinity, the random variables $\sqrt{n}(S_n - \mu)$ converge in distribution to a normal $\mathcal{N}(0, \sigma^2)$.

\subsection{Lists}

You can make lists with automatic numbering \dots

\begin{enumerate}
\item Like this,
\item and like this.
\end{enumerate}
\dots or bullet points \dots
\begin{itemize}
\item Like this,
\item and like this.
\end{itemize}

We hope you find write\LaTeX\ useful, and please let us know if you have any feedback using the help menu above.
\pagebreak


En la ecuación de la recta tangente $y = mx + b$ establezco las condiciones del problema:
\begin{itemize}
    \item Ordenada al origen $$b$$
    \item Abscisa al origen: raíz de la recta $$ 0 = mx + b$$
\end{itemize}
$$ x = -\frac{b}{m}$$
Entonces ahora busco que la recta tangente sea tal que \emph{la suma de la ordenada y la abscisa en el origen en un punto genérico sea igual a 2}:
$$b + (-\frac{b}{m}) = 2 $$
$$b (1-\frac{1}{m}) = 2$$
$$ \frac{m-1}{m} = \frac{2}{b} $$
$$ b = \frac{2m}{m-1} $$
Reemplazo en $y=mx+b$
$$ y = mx + \frac{2m}{m-1} $$
De donde se obtiene
\begin{equation}
\label{eq1:1}
    y(m-1)=m^2 x - mx + 2m
\end{equation}
Derivando 
$$ y'(m-1)=m^2 -m $$
$$ y'(m-1)=m(m-1) $$
\begin{equation}
\label{eq1:2}
    y' = m
\end{equation}
\ref{eq1:2} en \ref{eq1:1}
$$ y(y'-1)=(y')^2 x - y'x + 2y' $$
Luego
$$ x(y')^2-y'(x+y-2)+y = 0 $$

\end{document}