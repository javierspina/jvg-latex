\documentclass[9pt,a4paper]{extarticle}
\usepackage[utf8]{inputenc}
\usepackage[spanish]{babel}
\usepackage[top=0.25in, bottom=0.25in, left=0.4in, right=0.4in]{geometry}
\usepackage{amsmath, amssymb, amsfonts}
\usepackage{graphicx}

%letra en arial
%\usepackage{helvet}
%\renewcommand{\familydefault}{\sfdefault}

\usepackage{enumerate}% http://ctan.org/pkg/enumerate

\begin{document}
\pagenumbering{gobble}
\hrule
\center{Recuperatorio del Segundo Parcial de Análisis Matemático III - 4º C}
\center {3/10/2017}
\begin{enumerate}


    \item Representar graficamente los siguientes puntos del plano complejo
    \begin{enumerate}
        \item $2 \leq \Re(z) + \Im(\Bar{z}) < 5$
        \item $\Im(\overline{ \Bar{z} - z^{2} }) = 2 + \Im(z)$
    \end{enumerate}
    \item Transformar $y=2x-1$ mediante $w=\dfrac{z+1}{z-i}$ y graficar el conjunto de puntos y su transformado en planos superpuestos.
    \item Calular, si existe, el siguiente límite. $$\lim_{z \to 0} \dfrac{i \Re(z^2)}{|\Bar{z}|}$$
    \item Analizar la derivabilidad de las siguientes funciones, mediante las ecuaciones de Cauchy-Riemann; y, de ser derivables, hallar su derivada, utilizando dichas ecuaciones, expresada en variable $z$.
    \begin{enumerate}
        \item $w=ln(iz)$
        \item $w=2i\Re(z)$
    \end{enumerate}
    \item Sabiendo que $u=\Re[f(z)]/u(x;y)=3x^2 + x -3y^2$, halle --mediante las ecuaciones de Cauchy-Riemann-- $f'(z)$ (expresada en variable $z$), suponiendo a $f$ entera.
\end{enumerate}
\hrule

\center{Segundo Parcial de Análisis Matemático III - 4º C}
\center {5/09/2017}
\begin{enumerate}
\item Halle:
\begin{enumerate}
    \item el conjunto solución de la ecuación $e^{2z-i} = i$
    \item el conjunto de puntos que verifican la ecuación $\Im({\Bar{z}}^2) < 2$. Grafíquelo.
\end{enumerate}
\item Transforme $|z-1-3i|=\sqrt{5}$ mediante $w=\dfrac{2z-i}{iz+1}$ y grafique el conjunto de puntos y su transformado (en un mismo gráfico).
\item Calcule, si existe, el siguiente límite. $$\lim_{z \to 0} \dfrac{z-\Im(z)\Bar{z}}{|z|}$$
\item Analice la derivabilidad de las siguientes funciones, mediante las ecuaciones de Cauchy-Riemann; y, en caaso de ser derivables, deduzca la expresión de la derivada en variable z.
\begin{enumerate}
    \item $f(z)=\Re(z)i$
    \item $f(z)=e^{-3z}$
\end{enumerate}

\item Dada $v=\Im[f(z)]/v(x;y)=2xy+2x$, halle $w=f(z)$ (expresada en variable $z$) suponiéndola entera, sabiendo que $f(i)=-4$; y obtenga su conjuntos de ceros.
\end{enumerate}
\hrule

\center{Segundo Parcial de Análisis Matemático III - 4º C}
\begin{enumerate}
\item Grafique, en el plano complejo, los siguientes conjuntos de puntos.
\begin{enumerate}
    \item El $Df$, siendo $f(z)=\dfrac{z+4i}{\Re(z)-\Bar{z}}$
    \item El conjunto de puntos que verifican: $[\Re(z)]^2 \leq 2 \Im(\Bar{z})$
\end{enumerate}
\item Transforme $y=-2x+2$ mediante $w=\dfrac{-iz+1}{2z-2i}$ y grafique el conjunto de puntos y el de su transformado, en un mismo gráfico.
\item Analizar la continuidad de la siguiente función en el origen.
$$
     f(z) = \left\{
	       \begin{array}{cc}
		 \dfrac{\Re(z) \Im(z)i+\Bar{z}|z|}{z\Re(z)}      & \mathrm{si\ } z \neq 0 \\
		 &\\
		 0     & \mathrm{si\ } z = 0
	       \end{array}
	     \right.
$$
\item Analice la derivabilidad de las siguientes funciones, mediante las ecuaciones de Cauchy-Riemann; y, en caso de ser derivables, deduzca la expresión de la derivada y exprésela en variable $z$.
\begin{enumerate}
    \item $f(z)=z-\Re(z)$
    \item $f(z)=iz^3$
\end{enumerate}
\item Dada $u=\Re[f(z)]/u(x;y)=4x^2+3x-4y^2$, halle $w=f(z)$ (expresada en variable $z$), suponiéndola entera, sabiendo que $f(i)=-4+i$.
\end{enumerate}
\hrule
\newpage
\hrule

\center{Segundo Parcial de Análisis Matemático III - 4º C}
\begin{enumerate}
    \item Representar graficamente los siguientes puntos del plano complejo
    \begin{enumerate}
        \item $-1 < \Re(\Bar{z}) - \Im(z) \leq 4$
        \item $\Im(z) -4= \Im(\overline{ \Bar{z} - z^{2} })$
    \end{enumerate}
    \item Transformar $\Re(z)+\Im(z)-1=0$ mediante la aplicación $w=\dfrac{z-i}{z+1}$ y grafique el conjunto de puntos dado y su transformado, en planos superpuestos.
    \item Dada la función: $$f(z)= \dfrac{ \Re(z^2)+z}{|z|-\Bar{z}i}$$
    \begin{enumerate}
        \item Obtenga el conjunto $Df$.
        \item Calcule, si existe: $$\lim_{z \to 0} f(z)$$
    \end{enumerate}
    \item Analice la derivabilidad de las siguientes funciones, mediante las ecuaciones de Cauchy-Riemann; y, de ser derivables, halle la derivada, utilizando dichas ecuaciones, y exprésela en variable $z$.
    \begin{enumerate}
        \item $w=ln(2z)$
        \item $w=3i\Im(z)$
    \end{enumerate}
    \item Sabiendo que $v=\Im[f(z)]/v(x;y)=2xy-2x$, y que $f(0) = -1$: 
    \begin{enumerate}
        \item halle --mediante las ecuaciones de Cauchy-Riemann-- $f'(z)$ (expresada en variable $z$), suponiendo a $f$ entera.
        \item obtenga el conjunto $C^0 = {z \in \mathbb{C} / f(z)=0}$
    \end{enumerate}
\end{enumerate}
\hrule

\center{Segundo Parcial de Análisis Matemático III - 4º C}
\begin{enumerate}
\item Grafique, en el plano complejo, los siguientes conjuntos de puntos.
\begin{enumerate}
    \item El $Df$, siendo $f(z)=\dfrac{z+2i}{\Im(z)-|z|}$
    \item El conjunto de puntos que verifican: $\Im(\Bar{z})\geq 2\Re(z)$.
\end{enumerate}
\item Transforme $|z-1+i|=\dfrac{\sqrt{5}}{2}$ mediante $w=\dfrac{2iz+1}{2z+i}$ y grafique el conjunto de puntos y el de su transformado, en un mismo gráfico.
\item Analizar la continuidad de la siguiente función en el origen.
$$
     f(z) = \left\{
	       \begin{array}{cc}
		 \dfrac{\Bar{z}\Im(z)+\Re^2(z)i}{|z|^2}      & \mathrm{si\ } z \neq 0 \\
		 &\\
		 0     & \mathrm{si\ } z = 0
	       \end{array}
	     \right.
$$
\item Analice la derivabilidad de las siguientes funciones, mediante las ecuaciones de Cauchy-Riemann; y, en caso de ser derivables, deduzca la expresión de la derivada y exprésela en variable $z$.
\begin{enumerate}
    \item $f(z)=\Bar{z}+\Im(z)$
    \item $f(z)=(1+i)z^2$
\end{enumerate}
\item Dada $u=\Re[f(z)]/u(x;y)=3x^2-x-3y^2$, halle $w=f(z)$ (expresada en variable $z$), suponiéndola entera, sabiendo que $f(i)=-3+i$.
\end{enumerate}
\hrule

\center{Recuperatorio del Segundo Parcial de Análisis Matemático III - 4º C}
\center{27/9/2016}
\begin{enumerate}
    \item Representar gráficamente los siguientes puntos del plano complejo.
    \begin{enumerate}
        \item $1 \leq \Im(\Bar{z}) + \Re(z) < 3$
        \item $
     \left\{
	       \begin{array}{c}
		 |Arg(z)| < \dfrac{3}{4} \pi \\
		 \Re(z+3+i)>2\\
		 1 \leq |z| \leq 4
	       \end{array}
	     \right.
$
    \end{enumerate}
    \item Trnasformar $y=x$ mediante $w=\dfrac{z-1}{z+i}$ y graficar el conjunto de puntos y su transformado en planos superpuestos.
    \item Calcular, si existe, el siguiente límite $$ \lim_{z \to 0} \dfrac{i\Im(z^2)}{|z|}$$
    \item Analizar la derivabilidad de las siguientes funciones, mediante las ecuaciones de Cauchy-Riemann; y, de ser derivables, hallar su derivada, utilizando dichas ecuaciones, expresada en variable $z$
    \begin{enumerate}
        \item $w=e^{iz}$
        \item $w=\Bar{z}-\Re(z)$
    \end{enumerate}
    
    \item Hallar, si existe, la conjugada armónica de $u=2x-2xy$; y expresar $w=f(z)$, sabiendo que $f(i) = 1$.
\end{enumerate}
\hrule
\newpage
\hrule

\center{Segundo Parcial de Análisis Matemático III - 4º C}
\begin{enumerate}
\item Halle:
\begin{enumerate}
    \item el conjunto solución de la ecuación $e^{3z+i} = -i$
    \item el conjunto de puntos que verifican la ecuación $\Re({\Bar{z}}^2) > 1$. Grafíquelo.
\end{enumerate}
\item Transforme $\Im(z)=\dfrac{1}{2}\Re(z)-\dfrac{9}{4}$ mediante $w=\dfrac{z+i}{-iz+2}$ y grafique el conjunto de puntos y su transformado (en un mismo gráfico).
\item Calcule, si existe, el siguiente límite. $$\lim_{z \to 0} \dfrac{\Bar{z}+\Re(z)z}{|z|}$$
\item Analice la derivabilidad de las siguientes funciones, mediante las ecuaciones de Cauchy-Riemann; y, en caaso de ser derivables, deduzca la expresión de la derivada en variable z.
\begin{enumerate}
    \item $f(z)=\Im(z)i$
    \item $f(z)=e^{-2z}$
\end{enumerate}

\item Dada $u=\Re[f(z)]/u(x;y)=x^2-y^2-2y-1$, halle $w=f(z)$ (expresada en variable $z$) suponiéndola entera, sabiendo que $f(1)=2i$; y obtenga su conjuntos de ceros.
\end{enumerate}
\hrule

\center{Segundo Parcial de Análisis Matemático III - 4º C}
\center{5/8/2016}
\begin{enumerate}
    \item Represente gráficamente los siguientes conjuntos de puntos en el plano complejo.
    \begin{enumerate}
        \item $|z+3|+|z-3| \geq 10$
        \item $\Re(\overline{z^2-Bar[z]}) = 7-\Re(z-2)$
    \end{enumerate}
    \item Transforme $y=x+\dfrac{1}{4}$ mediante $w=\dfrac{-4iz-1}{2z-2i}$ y grafique el conjunto de puntos y su transformado (en un mismo gráfico).
    \item Estudie la continuidad de la siguiente función en $z_0=0$. $$
     f(z) = \left\{
	       \begin{array}{cc}
		 \dfrac{z^2+i\Bar{z}}{|z|}      & \mathrm{si\ } z \neq 0 \\
		 &\\
		 0     & \mathrm{si\ } z = 0
	       \end{array}
	     \right.
$$
\item Indique el dominio y analice la derivabilidad de $f(z) = ln |z|$
\item Dada $v + \Im[f(z)]/v(x;y)=2xy-y+2$:
\begin{enumerate}
    \item Halle $f'(z)$, utilizando las ecuaciones de Cauchy-Riemann y expresada en variable $z$.
    \item Para la función $f$ obtenida, prueve que $u(x;y)=c_1 y v(x;y)=c_2$ son familias de trayectorias ortogonales.
\end{enumerate}
\end{enumerate}
\hrule

\end{document}