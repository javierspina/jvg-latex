\documentclass[9pt,a4paper]{extarticle}
\usepackage[utf8]{inputenc}
\usepackage[spanish]{babel}
\usepackage[top=0.25in, bottom=0.25in, left=0.4in, right=0.2in]{geometry}
\usepackage{amsmath, amssymb, amsfonts}

%letra en arial
%\usepackage{helvet}
%\renewcommand{\familydefault}{\sfdefault}

\usepackage{enumerate}% http://ctan.org/pkg/enumerate

\begin{document}
\pagenumbering{gobble}
\center{ISP JVG - Análisis Matemático II - A. García Venturini}
\center {5 de julio de 2017 - Primer parcial}
\begin{enumerate}
    \item Dada
    \[
    \sum_{n=1}^{\infty}\dfrac{(-1)^n(x-2)^n}{3^{n+1}}
    \]
   determinar el intervalo de convergencia y si es posible determinar a que valor converge cuando $ x=3 $
    
    
    \item Dado el campo escalar, determinar
    \[
    f(x;y)= \dfrac{\sqrt{2x-x^2-y^2}}{\ln{(x+y-2)}} 
    \]

    \begin{enumerate}
        \item gráfica y analíticamente el dominio $D$
        \item el conjunto derivado
        \item si el dominio es un conjunto abierto o cerrado. Justificar.
    \end{enumerate}
    
    
    \item 
    \begin{enumerate}
        \item Analizar derivada direccional para todo $\alpha$ en el origen, dar la expresión
        \item Analizar continuidad, derivabilidad y diferenciabilidad en $ (0;0) $
        \item ¿Admite plano tangente? Justificar
    \end{enumerate}
    \[
    z= \left\{ \begin{array}{lcc}
             \dfrac{x^4 + y^4}{x^2 + y^2} &   si  & (x;y) \neq (0;0) \\
             \\ 0 &  si & (x;y) = (0;0)
             \end{array}
   \right.
    \]
    
    \item Analizar continuidad de 
    
        \[
    z= \left\{ \begin{array}{lcc}
             \dfrac{\sin{(x^2-y^2)}}{x- y} &   si  & x \neq y \\
             \\ 0 &  si & x = y
             \end{array}
   \right.
    \]
    
    
    \item Decidir si los siguientes enunciados son verdaderos o falsos. Justificar con presición.
    \begin{enumerate}
        \item Si el límite en coordenadas polares da $0$, entonces el límite doble vale $0$
        \item Si las derivadas parciales no son continuas, la función no es diferenciable.
    \end{enumerate}
\end{enumerate}
\hrule

\end{document}