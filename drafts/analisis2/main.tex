\documentclass[9pt,a4paper]{extarticle}
\usepackage[utf8]{inputenc}
\usepackage[spanish]{babel}
\usepackage[top=0.25in, bottom=0.25in, left=0.4in, right=0.2in]{geometry}
\usepackage{amsmath, amssymb, amsfonts}

%letra en arial
%\usepackage{helvet}
%\renewcommand{\familydefault}{\sfdefault}

\usepackage{enumerate}% http://ctan.org/pkg/enumerate

\begin{document}
\pagenumbering{gobble}
\center{ISP JVG - Análisis Matemático II - A. García Venturini}
\center {16 de noviembre de 2011 - Recuperatorio del primer parcial}
\begin{enumerate}
    \item Dada la serie 
    \[
    \sum_{n=1}^{\infty}\dfrac{(-1)^{n+1}}{n^{2}+1}
    \]
    \begin{enumerate}
    \item analizar si es convergente
    \item de serlo, determinar si la convergencia es absoluta o condicional
    \item hallar el valor aproximado de su suma si se considera la suma parcial de orden 4
    \item acotar el error que se comete
    \end{enumerate}
    
    
    \item Dado el campo escalar, determinar
    \[
    f(x;y)= \sqrt{36-x^{2}-4y^{2}} \ln{(x^2 + y^2 - 1)}
    \]

    \begin{enumerate}
        \item gráfica y analíticamente el dominio $D$
        \item los conjuntos $D_i$, $D_e$, $D_f$ y $D'$
        \item si el conjunto es abierto, cerrado, denso, perfecto. Justificar.
    \end{enumerate}
    
    
    \item Analizar continuidad y derivabilidad en el origen de la siguiente función. Justificar.    
    \[
    f(x;y)= \left\{ \begin{array}{lcc}
             \dfrac{x^3 + y^3}{x^2 + y^2} &   si  & (x;y) \neq (0;0) \\
             \\ 0 &  si & (x;y) = (0;0)
             \end{array}
   \right.
    \]
    
    \item Si denotamos con\[ h(x;y)=2e^{-x^2}+e^{-3y^2}\] la altura de una montaña en la posición (x;y). ¿En qué dirección y sentido desde (1;0) se debe comenzar a caminar para escalar lo más rápido posible?
    
    
    \item Decidir si los siguientes enunciados son verdaderos o falsos. Justificar con presición.
    \begin{enumerate}
        \item Una función derivable en un punto, es continua en ese punto.
        \item Las derivadas segundas cruzadas son iguales.
    \end{enumerate}
\end{enumerate}
\hrule
\center {3 de julio de 2013 - Primera evaluación parcial}
\begin{enumerate}
     \item Dada la serie 
    \[
    \sum_{n=1}^{\infty}\dfrac{(-1)^{n+1}}{4n+1}
    \]
    \begin{enumerate}
    \item analizar la convergencia absoluta y condicional
    \item aproximar por truncamiento de orden 4 la suma S
    \end{enumerate}
    
    
    \item Dado el campo escalar, determinar
    \[
    f(x;y)= \dfrac{\ln{(-x^2-y)}}{\sqrt{2-x^2-y^2}}
    \]
    
    \begin{enumerate}
        \item gráfica y analíticamente el dominio $D$
        \item determinar si el conjunto es perfecto. Justificar.
    \end{enumerate}
    
    
   \item Analizar en todo $ {\mathbb{R}}^2 $ la continuidad y derivabilidad de la siguiente función. Justificar.
   
    \[
    f(x;y)= \left\{ \begin{array}{lcc}
             \dfrac{x^3 + 4y^3}{x^2 + y^2} &   si  & (x;y) \neq (0;0) \\
             \\ 0 &  si & (x;y) = (0;0)
             \end{array}
   \right.
    \]
    
    \item Dado el campo escalar \[ f(x;y)=x^2y^3 \] 
    \begin{enumerate}
         \item demostrar, utilizando la definición, que es diferenciable en todo su dominio
         \item hallar la derivada direccional en $P_0 (2;1)$ en la dirección y sentido de la curva $ y=x^2-2x$ en $x_0=1$
         \item determinar dirección, sentido de la derivada direccional nula
         \item hallar por aproximación lineal $f(1,8;1,1)$
    \end{enumerate}
    
    
    \item Decidir si los siguientes enunciados son verdaderos o falsos. Justificar con presición.
    \begin{enumerate}
         \item La serie de potencias que se obtiene de integrar una serie de potencias, conserva el intervalo de convergencia
         \item Si el límite en coordenadas polares da cero, entonces el límite doble puede o no existir.
    \end{enumerate}
\end{enumerate}

\newpage
\center {6 de julio de 2011 - Primera evaluación parcial - Tema I}

\begin{enumerate}
     \item Dada la serie 
    \[
    \sum_{n=1}^{\infty}\dfrac{(-1)^{n+1}}{n^2+3}
    \]
    \begin{enumerate}
    \item analizar si es convergente
    \item de serlo, determinar si la convergencia es absoluta o condicional
    \item hallar el valor aproximado de su suma si se considera la suma parcial de orden 4
    \item acotar el error que se comete
    \end{enumerate}
    
    
    \item Dado el campo escalar, determinar
    \[
    f(x;y)= \dfrac{\sqrt{4-x^2-y^2}}{\ln{(x+y-2)}}
    \]
    
    \begin{enumerate}
        \item gráfica y analíticamente el dominio $D$
        \item los conjuntos $D_i$, $D_e$, $D_f$ y $D'$
        \item si el conjunto es abierto, cerrado, denso, perfecto. Justificar.
    \end{enumerate}
    
    
    
    \item Analizar la continuidad y derivabilidad de la siguiente función en el origen. Justificar.
   
    \[
    f(x;y)= \left\{ \begin{array}{lcc}
             \dfrac{x^3}{x + 5y} &   si  & x \neq -5y \\
             \\ 0 &  si & x = -5y
             \end{array}
   \right.
    \]
    
    \item Dado el campo escalar \[ f(x;y;z)=x^2+5xy+2z^2 \] 
    \begin{enumerate}
         \item hallar una derivada direccional en $P_0 (2;1;1)$ en la dirección paralela a la de la recta $(x;y;z)=(1+2\alpha;2-\alpha;1-3\alpha)$
         \item determinar dirección, sentido y valor de las derivadas direccionales máxima y mínima en $P_0$
         \item ¿Existe $\alpha / {f'}_{\alpha}(P_0)=20$? 
    \end{enumerate}
    
    \item Decidir si los siguientes enunciados son verdaderos o falsos. Justificar con presición.
    \begin{enumerate}
         \item El teorema de Schwarz establece que las derivadas segundas cruzadas son iguales
         \item Al derivar una serie de potencias, se conserva el intevalo de convergencia.
    \end{enumerate}
\end{enumerate}

\hrule
\center {4 de julio de 2012 - Primera evaluación parcial - Tema II}

\begin{enumerate}
    \item Encontrar el intervalo de convergencia de la serie
    \[
    \sum^\infty_{n=1} \dfrac{{(x+1)}^n}{n4^n}
    \]
    
    
    \item Dado el campo escalar, determinar
    
    \[
    f(x;y)= \dfrac{\arccos(x^2+y^2-3)}{\sqrt{x^2+y^2+4}}
    \]
    
    \begin{enumerate}
        \item gráfica y analíticamente el dominio $D$
        \item los conjuntos $D_i$, $D_e$, $D_f$ y $D'$
        \item si el conjunto es abierto, cerrado, denso, perfecto. Justificar.
    \end{enumerate}
    
    
    \item Analizar en todo ${\mathbb{R}}^2$ la continuidad, derivabilidad y diferenciabilidad de la siguiente función. Justificar.
    
    \[
    f(x;y)= \left\{ \begin{array}{lcc}
             \dfrac{x^4 + 4y^2}{x^2 + y^2} &   si  & (x;y) \neq (0;0) \\
             \\ 0 &  si & (x;y) = (0;0)
             \end{array}
   \right.
    \]
    
    \item Dado el campo escalar $f(x;y)=x^2+5xy-2y^2$, 
    \begin{enumerate}
        \item hallar la derivada direccional en $P_0 (2;1)$ en la dirección máxima y sentido hacia el punto $P (-1;5)$
        \item determinar dirección, sentido y valor de las derivadas direccionales máxima y mínima en $P_0$
        \item calcular por aproximación lineal $f(1,9;1,2)$
    \end{enumerate}
    
    \item Decidir si los siguientes enunciados son verdaderos o falsos
    \begin{enumerate}
        \item Si una serie alternada es condicionalmente convergente, entonces es convergente
        \item Un conjunto de puntos denso puede ser denso
    \end{enumerate}
    
\end{enumerate}


\newpage
\center {4 de julio de 2012 - Primera evaluación parcial - Tema I}


\begin{enumerate}
    \item Dada la serie
    \[
    \sum^\infty_{n=1} \dfrac{(-1)^{n+1}}{n^2+4n+2}
    \]
    \begin{enumerate}
        \item analizar si es convergente
        \item de serlo, determinar si la convergencia es absoluta o condicional
        \item hallar el valor aproximado de su suma si se considera la suma parcial de orden 4
        \item acotar el error que se comete
    \end{enumerate}
    
    \item Dado el campo escalar, determinar
    
    \[
    f(x;y)= \dfrac{\arcsin(x^2+y^2-3)}{\ln{(x^2+y^2+4)}}
    \]
    
    \begin{enumerate}
        \item gráfica y analíticamente el dominio $D$
        \item los conjuntos $D_i$, $D_e$, $D_f$ y $D'$
        \item si el conjunto es abierto, cerrado, denso, perfecto. Justificar.
    \end{enumerate}
    
    
    \item Analizar la continuidad, derivabilidad y diferenciabilidad de la siguiente función en el origen. Justificar.
    
    \[
    f(x;y)= \left\{ \begin{array}{lcc}
             \dfrac{x^3 - 2y^4}{x^2 + y^2} &   si  & (x;y) \neq (0;0) \\
             \\ 0 &  si & (x;y) = (0;0)
             \end{array}
   \right.
    \]
    
    \item Dado el campo escalar $f(x;y)=x^2y+4y^2x^2$, 
    \begin{enumerate}
        \item hallar una derivada direccional en $P_0 (-1;1)$ en la dirección del punto $P (3;5)$
        \item determinar dirección, sentido y valor de las derivadas direccionales máxima y mínima en $P_0$
        \item calcular por aproximación lineal $f(-0,9;1,2)$
    \end{enumerate}
    
    \item Decidir si los siguientes enunciados son verdaderos o falsos
    \begin{enumerate}
        \item Hay campos escalares de dos variables derivables que son continuos
        \item Una serie de potencias siempre converge a la imagen de una función
    \end{enumerate}
    
\end{enumerate}

\hrule
\center {6 de julio de 2011 - Primera evaluación parcial - Tema II}

\begin{enumerate}
     \item Determinar el intervalo de convergencia de la serie 
     \[ \sum^\infty_{n=0} {(-1)}^n \dfrac{{(3x-2)}^n}{5^n}
     \]
    
    
    \item Dado el campo escalar, determinar
    \[
    f(x;y)= \dfrac{\sqrt{1-x^2-\sfrac{y^2}{4}}}{\ln{(2x-y-2)}}
    \]
    
    \begin{enumerate}
        \item gráfica y analíticamente el dominio $D$
        \item los conjuntos $D_i$, $D_e$, $D_f$ y $D'$
        \item si el conjunto es abierto, cerrado, denso, perfecto. Justificar.
    \end{enumerate}
    
    
    
    \item Analizar la continuidad y derivabilidad de la siguiente función en el origen. Justificar.
   
    \[
    f(x;y)= \left\{ \begin{array}{lcc}
             \dfrac{x^2}{2x + y} &   si  & y \neq -2x \\
             \\ 0 &  si & y = -2x
             \end{array}
   \right.
    \]
    
    \item Dado el campo escalar \[ f(x;y;z)=2z^2-4xy+x^3 \] 
    \begin{enumerate}
         \item hallar una derivada direccional en $P_0 (1;2;-1)$ en la dirección paralela a la de la recta $(x;y;z)=(2+3\alpha;1-\alpha;2-2\alpha)$
         \item determinar dirección, sentido y valor de las derivadas direccionales máxima y mínima en $P_0$
         \item ¿Existe $\alpha / {f'}_{\alpha}(P_0)=15$? 
    \end{enumerate}
    
    \item Decidir si los siguientes enunciados son verdaderos o falsos. Justificar con presición.
    \begin{enumerate}
         \item Dado un campo escalar $ z=f(x;y) $ se verifica que $ {f'''}_{xxy} = {f'''}_{xyx}$
         \item Al integrar una serie de potencias, se conserva el intevalo de convergencia.
    \end{enumerate}
\end{enumerate}

\newpage
\center {7 de julio de 2010 - Primera evaluación parcial}

\begin{enumerate}
    \item Encontrar el intervalo de convergencia de la serie
    \[
    \sum^\infty_{n=1} \dfrac{{(x-2)}^n}{n3^n}
    \]
    
    
    \item Dado el campo escalar, determinar
    
    \[
    f(x;y)= \dfrac{\arccos(x^2+y^2-3)}{\sqrt{x^2+y^2-4}}
    \]
    
    \begin{enumerate}
        \item gráfica y analíticamente el dominio $D$
        \item los conjuntos $D_i$, $D_e$, $D_f$ y $D'$
        \item si el conjunto es abierto, cerrado, denso, perfecto. Justificar.
    \end{enumerate}
    
    
    \item Analizar en todo ${\mathbb{R}}^2$ la continuidad y derivabilidad  de la siguiente función. Justificar.
    
    \[
    f(x;y)= \left\{ \begin{array}{lcc}
             \dfrac{x^3 + {2xy}^2}{x^2 + y^2} &   si  & (x;y) \neq (0;0) \\
             \\ 0 &  si & (x;y) = (0;0)
             \end{array}
   \right.
    \]
    
    \item Dado el campo escalar $f(x;y)=x^2+5xy-2y^2$, 
    \begin{enumerate}
        \item hallar la derivada direccional en $P_0 (2;1)$ en la dirección máxima y sentido hacia el punto $P (-1;5)$
        \begin{enumerate}[(i)]
            \item por definición
            \item por vector gradiente
        \end{enumerate}
        \item determinar dirección, sentido y valor de las derivadas direccionales máxima y mínima en $P_0$
    \end{enumerate}
    
    \item Decidir si los siguientes enunciados son verdaderos o falsos. Justificar con presición
    \begin{enumerate}
        \item Si una serie alternada es condicionalmente convergente, entonces es convergente
        \item Un conjunto de puntos perfecto ser denso
    \end{enumerate}
    
\end{enumerate}

\end{document}