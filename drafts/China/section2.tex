En China hay dos grandes filósofos: Lao-Tse y Confucio. Es una filosofía que se basa en la contemplación, a diferencia de los griegos que tenían una formulación activa de preguntas. Ellos esperan a que lleguen las respuestas (o las preguntas). Eso hace que todo sea mucho más lento. 

Es una cultura que se basa en el respeto a dos cosas: al orden (social, político, se sospecha que se basa en la manera cruel en que fueron ordenados como un imperio) y los antepasados. Esto último tiene una consecuencia importante en la historia de las ciencias. Tenían la idea de que si se tenía un teorema que se demuestra y es atribuido a quien lo demuestra, se está faltando el respeto a todos los que antes no pudieron hacerlo. Entonces cuando lograban un resultado nuevo lo que solían hacer era atribuirlo a algún antepasado. Esto complica mucho el trabajo del historiador: un resultado quizá estaba atado 200 años antes de cuando ocurre, es decir, no se puede datar de cuando son las obras.