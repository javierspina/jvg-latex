En cuanto a Aristóteles, su aporte a la matemática y a la ciencia corresponde a cómo se organiza el pensamiento científico. En la época de Aristóteles ya estaba bastante avanzado lo que es la democracia y dar un discurso para convencer al otro (retórica y dialéctica). Aristóteles piensa que la ciencia no tiene que pasar por eso, dice que puede pasar que se puede convencer a alguien de algo que no sea la verdad. 

Diferencia el conocimiento como Logo (ciencia, regido por la verdad) y Doxa (opinión, puede estar manejada por dialéctica y retórica). Entonces con estas premisas dice que la ciencia tiene que siempre llegar a enunciados de propiedades que son verdaderas. ¿De qué se parte entonces? De enunciados que sean verdaderos y tener una manera de ir deduciendo hasta llegar a otros enunciados que son verdaderos. 

Se presenta el problema de cómo saber si los enunciados de los que se parte son verdaderos. Aristóteles dice que estos son los enunciados que tienen que ser evidentes. Esa propiedad de ser evidentes se basa en el intuicion, que según él es la forma más primitiva de la razón. Una vez que tenga esos puntos de partida que son evidentes uso la lógica para llegar a otras proposiciones que si se deducen lógicamente, van a mantener las propiedades de las primeras que eran verdaderas. De esa manera voy se deshace la posibilidad de engaño y se evitan demostraciones circulares. 

Se tienen que establecer ciertos puntos de partida, que los va a llamar nociones comunes (porque son comunes a todas las ciencias) y postulados (propios de la disciplina que estoy estudiando). Las nociones comunes y los postulados tienen la característica de ser evidentes, no necesitan ser demostrados. Por intuición se puede ver que son verdaderos, evidentes. 

Por ejemplo una de las nociones de Euclides es “el todo es mayor que las partes”, “si a cosas iguales le sumo (o quito) cosas iguales obtengo cosas iguales” o “dos cosas iguales a una misma cosa son iguales entre sí”. Son cosas evidentes y además valen para todo ámbito, no solo matemática. Los postulados serán propios de la doctrina que se estudia. Partiendo de nociones comunes y a través de la lógica deductiva se obtienen nuevas proposiciones, propiedades que son los teoremas. Algo similar hace en relación a las definiciones, dice que se tiene que tener ciertos términos que no defino, porque si no se cae en demostraciones circulares: se tiene que tener términos primitivos que no se definen y a partir de ellos se definen nuevos términos. Esa estructura para la ciencia, quien la va a llevar a la práctica es Euclides. 
