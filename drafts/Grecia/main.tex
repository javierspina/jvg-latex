%-------------------------%
%-----Document Setup------%
%-------------------------%
\documentclass[a4paper,12pt,openany]{article}
\usepackage[utf8x]{inputenc}
\usepackage[spanish]{babel}
\usepackage[left=3cm, right=2cm, top=2.5cm, bottom=3cm]{geometry}
%\usepackage{savetrees}
\usepackage{verbatim}
\linespread{1.6}
%\setlength{\footskip}{20pt}
\usepackage{graphicx}
\usepackage{enumitem}


%-------------------------%
%------Document Code------%
%-------------------------%
\newcommand{\thought}[1]{\textit{#1}}

\newcommand{\scenechange}{
  \par
  \vspace{\baselineskip}
  \par
\noindent}
%Creates a line break for a change of scene

\newcommand{\majorchange}{
  \par
  \vspace{\baselineskip}
  \hfill
  \textasteriskcentered
  \hfill
  \vspace{\baselineskip}
\noindent}
%creates a major line break, split by an asterisk for scene changes at the end of a page of where a sense of a major change is required. 

%-------------------------%
%------Main Document------%
%-------------------------%
\begin{document}
\title{\vspace{-2cm}Grecia}
\date{}
\maketitle

\section{Introducción: La razón y el pensamiento en Grecia}
Los árabes estuvieron ubicados en un territorio extenso, comienzan su época de viajes a partir del 622 d.C., que es el momento en el que Mahoma deja la ciudad de la Meca. En ese momento los árabes empiezan a viajar tratando de convertir a los distintos pueblos con los que iban teniendo contacto y a su vez, trataban de absorber la cultura que encontraban y traducirla a su lengua, ya que consideraban que todo debía ser traducido al árabe para después trabajar en esos conocimientos (científicos, matemáticos, lo que fueran). 

Se trata de un pueblo viajero, que lo hace desde el norte de la India hasta el sur de España: todo esto fue una zona de influencia árabe. Su forma de orientación era a través de la astronomía. Su astronomía es producto de tres culturas: Siria, Babilonia e India. Le dan mucho valor a la lectura, es obligación leer el Corán (es la única religión en donde el fundador no era analfabeta).


\section{Período Clásico (Helénico)}
El faraón cobraba impuestos directamente proporcionales al tamaño de la tierra. Eso generó algunos conocimientos matemáticos: proporción, que estaba sujeta a cambios, por ejemplo si desbordaba el nilo, la cantidad de tierra cosechada era menor por lo tanto había que pagar menos. Y por el caso contrario si un terreno bien sedimentado producía una buena cosecha, se pagaba más. Por lo que había que estar constantemente midiendo tierras. 
Tenían un sistema de numeración, era no posicional. Y también unidades de medida para longitudes, superficies. Las unidades correspondían a las medidas del faraón. Palmo, codo, pie. Esto traía problemas porque a veces los faraones eran niños, o si cambiaba el faraón, cambiaba el patrón de unidad. Luego crean medidas estándar (que habrán provenido de algún faraón) que se mantuvieron constantes. 

Los impuestos se pagaban con semillas. Para saber cuánto pagar, necesitaban unidades de capacidad. Tenían vasijas de distintos tamaños, unas múltiplos de las otras.

Necesitaban alguna especie de calendario para contar con alguna manera de predecir las crecidas del Nilo. 

Tenían una cuerda de 12 nudos para medir ángulos rectos, esto directamente relacionado con la necesidad de parcelar las tierras para la agricultura. Por ejemplo, si tengo una cuerda con 12 nudos y los dispongo en lados imaginarios compuestos por 3, 4 y 5 nudos tengo triángulo rectángulo, un triángulo que verifica el teorema de pitágoras. Había egipcios que tenían como trabajo medir ángulos rectos: eran los tendedores de cuerdas. De esta manera descubrieron que 3, 4 y 5 es lo que llamamos una terna pitagórica: números que verifican el teorema de pitágoras. Y descubrieron propiedades al respecto: que si a cada número lo multiplicaban por otro número, seguía siendo una terna pitagórica.

El sistema de numeración es un sistema no posicional que permitía representar enteros no positivos y también números racionales positivos. No existen los números negativos para ellos y tampoco tienen cero. 
\\[0.5cm]
        \begin{minipage}{0.2\textwidth}
            \begin{flushleft} 
\includegraphics[scale=0.2]{20160804_080346.png}
            \end{flushleft}
        \end{minipage}
        ~
        \begin{minipage}{0.74\textwidth}

A pesar de no tener un cero, tenían un protocero, "nfr". Tiene algunas características del cero pero no simboliza la nada. El símbolo se encontraba en las pirámides, dispuesto de forma tal que extendido hacía fuera de la pirámide marcaba el nivel del suelo. Del cero para abajo hay tierra, del cero para arriba hay aire. Es como una especia de nivel, un origen de coordenadas. Pero no solo se encontraba en las pirámides. En los libros contables del faraón aparecía cuando estaba equilibrado el saldo a favor y el saldo en contra. Simbolicamente representa la belleza y el equilibrio.

        \end{minipage}\\[0.5cm]

Aunque para los sistemas no posicionales es difícil hacer algoritmos, ellos desarrollan un algoritmo de multiplicación que involucra propiedades asociativas, distributivas. 

Resolvían problemas de geometría del espacio: el volumen de una pirámide, áreas laterales de la pirámide; también volumen de una semiesfera para saber el volumen de unas cestas que tenían, también la superficie de la semiesfera, para saber la superficie que tenían que tejer. Trabajaron también con círculos, calculaban área y longitud de la circunferencia. Utilizaron varios valores para pi, pero el que más usaban era 22/7 (3,1428, bastante aproximado). 

Resolvían ecuaciones de 1er y 2do grado, y si bien tenían algoritmos para multiplicación y suma, tenían las operaciones tabuladas. Era mucho más rápido mirar en una tabla que hacer cuentas. También tenían concepto de sucesiones y series como concepto de sumas repetidas, no llegaban al infinito (no lo manejaban conceptualmente) pero si tenían la noción de poder ir agregando términos. No se plantean su naturaleza de entero o racional o si la suma sucesiva tiende a algo, es algo mucho más concreto (es una cultura empírica, no tiene demostraciones. para que algo este bien basta con que la práctica lo corrobore).









\subsection{Thales de Mileto}
El segundo movimiento es el Jainismo, que es un movimiento filosófico que no tiene origen religioso. Se basa en 2 principios. Uno de ellos es la pluralidad de ideas. Ellos aceptan que dos personas pueden pensar distinto y no por eso una de las dos está equivocada. No existe en el jainismo la lógica bivalente, sino que sus lógicas son polivalentes (en el vedismo usan lógicas bivalentes en sus demostraciones). Para ellos tiene razón el que convence mejor al otro: el que tiene mayor habilidad para persuadir es quien tiene argumentos más sólidos; no es una cuestión de contenido sino de cómo lo diga. El segundo principio es el respeto a esa pluralidad. Si considero que dos ideas distintas pueden ser correctas, entonces tengo que respetar ambas. En determinado momento trabajan con una lógica de cuatro valores, después llegan a siete valores. Su lógica al no ser bivalente hay algunos principios aristotélicos que no valen, por ejemplo el tercero excluido, el principio de no contradicción. Estos dos principios fundantes de la lógica occidental, no son válidos con lógicas polivalentes. 

El hecho de basarse en estos dos principios, hace que el pensamiento jainista cuando llega a una contradicción, no sea fuente de problemas. Por ejemplo el infinito, en el caso de Grecia, traía contradicciones y lo dejaron por fuera del análisis. Para los jainistas no es un problema pensar en el infinito. Entonces, comienzan a preguntarse cuántas almas existen: ya que creen en la reencarnación, dicen que simultáneo hay muchas almas en el universo, algunas reencarnadas en la Tierra, otras en otros planetas, otros lugares. Ante la pregunta de cuántas hay, llegan a responder que existen $2^{96}$ almas. 

Se hacen otro tipo de preguntas que involucran números muy grandes, por ejemplo si los dioses son muy rápidos cuando vuelan, ¿qué distancia pueden viajar en el tiempo de un parpadeo? Entienden que son distancias muy grandes y crean una unidad para eso que es lo que llaman la Shojana. Luego, se plantean un cubo que tenga una shojana de arista que se llena con vellones de corderos, y se preguntan cuánta es la cantidad de vellones. 

Este tipo de interrogantes los lleva a empezar a clasificar los números. Reconocen los números contables, los que se pueden llegar a contar; otra categoría son los números incontables que son tan grandes que incluso se pierde la noción de cuántos son. Otro tercer tipo de números son los infinitos, números aún más grandes que esos números incontables. Es decir, este reconocimiento de números reconoce al infinito como un número y también no tienen el temor de incluirlo en el pensamiento matemático. Siendo un número, se puede operar. 

La idea de infinito entonces, surge en India y la llaman Ananta. En la mitología de la India, Ananta es una serpiente que es guardiana de las puertas del infierno y solía ubicarse en un lago sagrado, y Visnú se sentaba arriba de ella para pasear por el lago. La serpiente se ponía en tal posición que termina desembocando en el símbolo del infinito. Si bien el que se usa hoy en día es el mismo, no tiene el mismo origen (que se debe a un matemático Inglés que no se sabe si tenía conocimiento de la matemática de la India).

Además del infinito, tienen el cero. Surge con el nombre de Sunya: en medio de todas las preguntas que se hacen, se empiezan a preguntar por el no-ser. Descubren que hay distintos tipos. Para ellos no es lo mismo no existir que no estar presente, no ser pensado, no haber sido concebido, etc. Identifican más de 20 tipos de no existencia. Una de las formas es la nada. Cuando identifican la nada necesitan simbolizarla de alguna manera y utilizan un pequeño círculo que recibe el nombre de Sunya. 

Este cero tiene tres funciones, una es la representación de la nada (es la primera cultura que necesita nominarla). Otra función es posicional: el sistema de numeración de la India es el heredado actualmente, tiene la misma función. Y la tercera es una función operatoria, que corresponde a ciertas operaciones que se hacen distintas si está presente el cero, como hoy en día. Por ejemplo la división por cero, cuando hay una potenciación a la cero que es igual a 1. El cero de la India es el primero en tener estas tres funciones, que hemos heredado.


\subsection{Pitágoras de Samos}
El tener el cero dió origen a un sistema de numeración decimal posicional, con muchas de las características del sistema decimal actual: números racionales, negativos, positivos, enteros. Los que llevan a Europa este sistema son los árabes, por eso es conocido como sistema indo-arábigo: es creado en India y adoptado y divulgado por los árabes. 

\subsection{Matemática y sociedad}
Hasta aquí hemos visto a los principales matemáticos de esta época en grecia, que demuestran gran avance respecto de las otras culturas. Sin embargo, la matemática griega tiene puntos muy bajos, como por ejemplo su sistema de numeración. Ninguno es posicional. El tratar de hacer un cálculo en cualquiera de sus sistemas es absolutamente tedioso. 

¿Qué hacían entonces? Utilizaban esclavos para hacer cuentas. La sociedad griega es terriblemente esclavista. Para hacer cuentas usaban esclavos babilónicos. Para ellos, hacer cálculos era una tarea denigrante. Así, nunca necesitaron un sistema de numeración decente pues ellos no hacían las cuentas. De hecho utilizaban a los esclavos para cualquier tarea que no era digna de la razón.

\subsection{Sócrates}
al-Kashi es un matemático que también trabaja en el área de astronomía, analiza mucho los eclipses lunares, hace un estudio y puede predecirlos, explicarlos, realiza tablas de los eclipses. Logra para el número Pi 16 cifras decimales con el método de Arquímedes, trabajando con un poliedro de 800 lados, trabajado teóricamente. Propone algoritmos para raíces enésimas, que son muy complicados, al ser generalizaciones de la solución para la raíz cuadrada.

\subsection{Platón}
En el caso de Platón, los objetos matemáticos tienen existencia en el mundo de las ideas. Lo que vemos alrededor nuestro son reflejos de estos objetos. No se tiene un rectángulo, se tiene una hoja con forma rectangular, que refleja lo que el rectángulo es en el mundo de las ideas. También son parte de su obra la alegoría de la caverna, que tiene que ver como nosotros accedemos el mundo de las ideas: meditando, filosofando, preguntandonos cosas. Según Platón, lo más cercano a las ideas es la matemática porque es la disciplina o ciencia que trabaja con objetos abstractos, por lo tanto está trabajando con el mundo de las ideas.

\subsection{Aristóteles}
El primero fue Aryabatha, que fue un matemático que propuso algoritmos para extraer raíces cuadradas y cúbicas. Trabaja con una aproximación del número $\pi$ de 3,1416. Es el creador de las primeras funciones trigonométricas. Las trabaja sobre un círculo, pero por ejemplo el seno no es cateto opuesto sino arco sobre hipotenusa. Esta función que propone se trabaja igual que el seno, tiene sus mismas características y las mismas propiedades. Hace tablas de esta función, similares a las actuales.

\section{Período Alejandrino (Helenístico)}
Brahmagupta trabajó conocimientos de astronomía en mayor medida y necesitó diseñar calendarios para medir el tiempo, hacer cuentas con números muy grandes. En determinado plantea una ecuación diofántica $61x^2 + 1 = y^2$ que tiene como soluciones más pequeñas a $x=226153980$ e $y=1766319049$. Usó, antes del siglo X aproximadamente, lo que hoy conocemos como ecuaciones de Pell, que fueron resueltas en occidente recién en el siglo XVII.

\subsection{Euclides}
A Bhaskara se le atribuye la ecuación resolvente, o más específicamente, el reconocimiento de las dos raíces (la fórmula como procedimiento ya existía de antes). También se le atribuye el volver, o el utilizar las primeras demostraciones deductivas que están plasmadas en los Sulva Sutras; si bien utilizó demostraciones gráficas, fue uno de los primeros en utilizar demostraciones deductivas con lógica bivalente. También trabaja con series y sucesiones. De hecho, él no es el único: en la India hay mucho trabajo con sucesiones y series. 

El uso del infinito como un número los lleva a tener series que todas son convergentes, aunque converjan al infinito. Bhaskara opera muchas veces con el cero y con el infinito, y no le genera ningún tipo de problemas, ya que para la matemática de la India eran todos números operables. También reconoce que la raíz cuadrada de un número va a tener siempre dos resultados, positivo y negativo. Esto lo lleva a concebir que la cuadrática tenga dos soluciones.


\subsection{Los temores griegos. Infinitos}
Los dos temores nacen a partir de las paradojas de Zenón. Zenón es del período helénico, que lo que plantea son tres paradojas: la más conocida es la de Aquiles y la tortuga: deciden correr una carrera, él era el griego más veloz, entonces Aquiles le cede a la tortuga que empiece con una ventaja, por ejemplo de un metro. Entonces cuando Aquiles llega a donde estaba la tortuga, ésta ya había avanzado nuevamente. Cuando Aquiles llegue a este nuevo lugar, la tortuga va a haber avanzado otro poco y así sucesivamente. Quiere decir que nunca la alcanza, pero en la práctica Aquiles sería el gran ganador, entonces ¿dónde está la contradicción?. Después de mucho discutir dicen que está involucrado el infinito por tener que hacer infinitos pasos y también el movimiento. De aquí los dos temores griegos. 

Se dan cuenta que con su pensamiento racional no pueden manejar al movimiento y al infinito. Entonces, Aristóteles va a decir que tenemos 2 tipos de infinito: el potencial y el actual. El infinito potencial es la capacidad que tienen algunos objetos matemáticos de seguir creciendo. Por ejemplo si se piensa en los números naturales: si yo pienso en un número natural muy grande siempre puedo encontrar un número mayor. Es decir, puedo seguir haciéndolo crecer. Hoy en día se piensa en todos los naturales todos juntos. Ellos los pensaban de manera acotada sabiendo que tienen la posibilidad de ampliarlos. 

Otro ejemplo de este uso del infinito es como ve Euclides las rectas. En los Elementos, se encuentra que cuando se tiene una recta se piensa en una recta AB, y por ejemplo luego dice de prolongar la recta AB hasta el punto C. Luego hasta el punto D… en realidad lo que está haciendo es trabajar con segmentos que se pueden seguir prolongando. Estos ejemplos son del infinito potencial, la posibilidad de seguir creciendo. No se piensa en un infinito actual como se piensa hoy en día, hoy se toma la recta y decir que se prolonga la recta suena raro. Se puede prolongar un segmento dentro de la recta, pero la recta ya es un elemento infinito. 

Los griegos no se meten con este infinito actual, no piensan a la recta en su totalidad: los piensan con la característica de que pueden seguir creciendo.


\subsection{Diofanto}
Otro matemático de Alejandría es Diofanto. Él tiene una obra que se llama Aritmética, donde describe cómo resolver las ecuaciones que hoy conocemos como diofánticas: ecuaciones con coeficientes naturales o enteros y sus soluciones son números enteros.

\subsection{Apolonio}
Tiene una obra que se llama Las Cónicas, ahí describe cómo desde un doble cono se puede seccionar con planos y obtener lo que se conoce hoy en día como cónicas: elipse y circunferencia, hipérbola, parábola. Apolonio las estudia desde un punto de vista geométrico (no algebraico como hoy en día con matrices). En esa época se conciben aparatos con engranajes para construir cónicas.

\subsection{Arquímedes de Siracusa}
En Siracusa, gobernaba un tirano, Hierón que le encomendó a Arquímedes un problema, el de la corona que había mandado a hacer y desconfiaba de su composición, creía que el joyero lo había estafado y le había colocado más plata que oro de la solicitada. Entonces Arquímedes, solucionando esto, postula que todo cuerpo sumergido en un líquido recibe un empuje de abajo hacia arriba igual al peso del líquido desalojado. 

Además, calcula el volumen de una esfera. Para lograrlo, la corta en fetas o cintas de muy poco espesor, de manera que se pueda pensar a las fetas como un cilindro, entonces a cada cinta le calculó su volumen, como un cilindro y después las sumó todas. Esto es un antecesor de las integrales, que se comenzaron a utilizar en el siglo XVI. Arquímedes usa nociones parecidas a las de integrales, diferenciales, idea de infinito (porque esas secciones tenían que ser infinitas, es muy sugerente que no le tenía miedo al infinito). Cabe destacar que su pensamiento es muy distinto al pensamiento de los otros griegos. De hecho estas ideas no son valoradas y no van a ser utilizadas o retomadas por matemáticos de esa época.

Otra introducción al infinito que Arquímedes hace es tener una circunferencia, y dentro de esta inscribir (o circunscribir) polígonos: triángulo, cuadrado, pentágono… mientras más lados se tenga, más se va a acercar a la circunferencia. Entonces estos pequeños perímetros sumados se acercan al perímetro de la circunferencia. Llegan a la conclusión de que el perímetro es el diámetro por una constante, que todavía no se llama Pi (Euler la va a llamar así). Él propone encontrar esta constante con este método. Esto lo retoman posteriormente, por ejemplo en la cultura árabe.


\subsection{Los tres problemas clásicos}
En esa época descubren que una manera de construir figuras es con regla y compás. Entonces creen que todo se puede construir con regla y compás. Plantean que, dado un círculo, se quiere construir un cuadrado que tenga la misma área. Esto se llama la cuadratura del círculo. Otro problema que plantean es duplicar el volumen de un cubo: la duplicación del cubo. El tercer problema que plantean es la trisección del ángulo, de la misma manera que se forma la bisectriz. Son problemas que muchos siglos después se demuestran que no se pueden resolver: el primer problema involucra al número Pi que no es construible con regla y compás. El segundo, ecuaciones cúbicas que tampoco se resuelven con regla y compás. El tercero vuelve a involucrar números irracionales que no son construibles.

\section{Período Grecorromano}
El tercer período es el período grecorromano, de decadencia. Son invadidos por los romanos cuya visión sobre la matemática y la ciencia es totalmente distinta. Para los romanos solo vale todo aquello que tenga una aplicación concreta, todo aquello que sirve para construir algo. No son empíricos, más bien pragmáticos. 

Sus construcciones son monumentales y duraderas: acueductos, bóvedas, arcos. Para esto necesitan mucho cálculo. Lo que no sirve para este fin no es tenido en cuenta. Es decir, toda la búsqueda griega por construir cosas solo con regla y compás o la organización axiomática de la matemática no tiene valor. Lo poco que rescatan por ejemplo es: las proporciones del ser humano, son tenidas en cuenta en sus esculturas y artes. También resultados útiles para ellos como el teorema de Pitágoras. Les llamaba la atención y rescataron de este teorema los resultados con irracionales.

Vitruvio hace una recorrida y un listado de las cosas que se rescatan de Grecia: esto es enmarcado en un tratado de Arquitectura: proporciones, Pitágoras (el resultado de los dos catetos iguales a 1). Para Roma la mayor parte de la cultura griega no tiene importancia, por esto este período es considerado de decadencia.



\end{document}