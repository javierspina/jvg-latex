%-------------------------%
%-----Document Setup------%
%-------------------------%
\documentclass[a4paper,12pt,openany]{article}
\usepackage[utf8x]{inputenc}
\usepackage[spanish]{babel}
\usepackage[left=3cm, right=2cm, top=2.5cm, bottom=3cm]{geometry}
%\usepackage{savetrees}
\usepackage{verbatim}
\linespread{1.6}
%\setlength{\footskip}{20pt}
\usepackage{graphicx}
\usepackage{enumitem}


%-------------------------%
%------Document Code------%
%-------------------------%
\newcommand{\thought}[1]{\textit{#1}}

\newcommand{\scenechange}{
  \par
  \vspace{\baselineskip}
  \par
\noindent}
%Creates a line break for a change of scene

\newcommand{\majorchange}{
  \par
  \vspace{\baselineskip}
  \hfill
  \textasteriskcentered
  \hfill
  \vspace{\baselineskip}
\noindent}
%creates a major line break, split by an asterisk for scene changes at the end of a page of where a sense of a major change is required. 

%-------------------------%
%------Main Document------%
%-------------------------%
\begin{document}
\title{\vspace{-2cm}Grecia}
\date{}
\maketitle

\section{Introducción: La razón y el pensamiento en Grecia}
Las culturas previas a Grecia eran empíricas. En la cultura griega hay un cambio de paradigma sobre qué es la ciencia y qué es la matemática, en ese cambio pasamos a dejar de tener lo que se llaman explicaciones míticas de los fenómenos y se pasa a tener explicaciones racionales. El griego busca explicar la realidad y todo lo que los rodea a través de la razón. Para los griegos, la razón es la que va a regir todo tipo de conocimiento y se va a constituir como algo característico del ser humano. 

A diferencia de otras culturas donde se buscaban resultados por ensayo y error, los griegos apuntan a generar una sistematización de conocimientos, una generalización de resultados. Buscan propiedades atribuibles a un conjunto en vez de ir por los casos particulares. Aparecen las demostraciones deductivas, una vez efectuada la demostración se va a saber para qué casos aplica.

Se distinguen tres períodos

\begin{enumerate}[topsep=0pt,itemsep=-1ex,partopsep=1ex,parsep=1ex]
    \item Clásico o Helénico: surgimiento del pensamiento racional (600 a.C. - 300 a.C.)
    \item Helenístico o Alejandrino: apogeo (300 a.C. - 300 d.C.)
    \item Grecorromano: decadencia (300 d.C. - 600 d.C.)
\end{enumerate}

\section{Período Clásico (Helénico)}
En el período helénico se sientan las bases del pensamiento griego. Surgen los llamados pensadores presocráticos, que buscan el principio de todas las cosas, en el intento de la explicación racional, el llamado arjé. A partir de esta búsqueda es que dan explicaciones de los distintos fenómenos que encuentran. Thales de Mileto y Pitágoras de Samos son de este período.

\subsection{Thales de Mileto}
Para Thales, el arjé es el agua, dice que cualquier cosa del universo está compuesta de agua. A Thales se le atribuyen las primeras demostraciones racionales, esas demostraciones donde se enuncia un teorema y se trata de demostrarlo con el pensamiento deductivo. Sin embargo, en algunas de sus demostraciones incurren en lo que se denominan demostraciones circulares. Es decir, que para demostrar una propiedad B utiliza una propiedad A y para demostrar A vuelve a utilizar B.

Algo que le dio mucho prestigio a Thales fue haber predecido el primer eclipse. En su búsqueda de tratar de explicar fenómenos, empezó a tener un modelo del movimiento del Sol, la Tierra y la Luna. Luego logró explicar un eclipse de Sol, diciendo que la Luna se ubica entre la Tierra y el Sol, además de poder predecirlo para el 28 de mayo del 585 a.C.. Esta predicción fue afamada y le dió prestigio porque el eclipse se dió durante una batalla contra los macedonios: el enemigo quedó confundido por el fenómeno, pero los griegos al saber del rumor, pudieron no distraerse tanto y así derrotaron a los macedonios fácilmente. La credibilidad que ganó Thales permitió que su modelo astronómico sea respetado. Fue incluido en varios listados de siete sabios griegos (de hecho, fue el único presente en todos los listados).

Otro modelo que tiene Thales es de los imanes. Él explica que dentro de un imán hay unos pequeños imanes. Por ejemplo con el fenómeno de la imantación del hierro: dentro de este material los pequeños imanes están desordenados, pero al estar en contacto con un imán, esos imanes del hierro se van ordenando, obteniendo así, un imán artificial. Es un modelo similar al actual. Decía que los pequeños imanes están a nivel atómico. De hecho, la palabra átomo viene de esta época, de Demócrito. Demócrito explica a la materia, dice que cualquier elemento que tenga lo puedo dividir sucesivamente hasta llegar a un punto que no pueda seguir dividiendo porque llego a algo elemental que no tiene partes. Átomo es eso, aquello que no tiene partes, la menor división posible de la materia. En el modelo de Thales está esta idea de átomo, los pequeños imanes están en este nivel.

Otro aporte de Thales es el Teorema de las rectas paralelas. Una aplicación que hace es para medir la altura de la pirámide de Keops. Para esa época resultaba muy difícil medir una pirámide. Thales visita Egipto, donde le plantean el problema de medir la pirámide. Él clava un bastón y observa las sombras del bastón y la pirámide. Con los datos de las sombras y mediante un esquema, plantea una proporcionalidad de los lados.

\subsection{Pitágoras de Samos}
Otro matemático árabe es Tabib ibn Quna, que es descendiente de sacerdotes babilónicos, y es uno de los principales traductores de los árabes, traduce todas las obras que encuentra de babilonio a árabe, pero también de otros idiomas, muchas desde el griego. Las traducciones no son solo de matemática sino que de medicina, botánica. Las traducciones que hace de matemática no son literales, sino que son comentadas, por ejemplo traduce los elementos de Euclides y va comentando si está de acuerdo o no con la demostración si le gusta, si lo hubiera hecho de otra forma, intenta probar el 5to postulado; y agregando aportes propios. 

Entre estos aportes, hay una generalización del teorema de Pitágoras: no solo construir cuadrados con los lados, sino cualquier figura, mientras sean semejantes, se mantiene el teorema. También hace una traducción de las cónicas de Apolonio, obras de Arquímedes (de hecho, muchos de los textos que llegan a nosotros llegan a través de estas traducciones árabes). También propone un método para medir longitudes de parábolas: dividía la parábola en pequeños segmentos y los rectificaba y de esa manera llega a medir la longitud de un arco de parábola. En occidente lo va a trabajar de forma similar Cavalieri.


\subsection{Matemática y sociedad}
Hasta aquí hemos visto a los principales matemáticos de esta época en grecia, que demuestran gran avance respecto de las otras culturas. Sin embargo, la matemática griega tiene puntos muy bajos, como por ejemplo su sistema de numeración. Ninguno es posicional. El tratar de hacer un cálculo en cualquiera de sus sistemas es absolutamente tedioso. 

¿Qué hacían entonces? Utilizaban esclavos para hacer cuentas. La sociedad griega es terriblemente esclavista. Para hacer cuentas usaban esclavos babilónicos. Para ellos, hacer cálculos era una tarea denigrante. Así, nunca necesitaron un sistema de numeración decente pues ellos no hacían las cuentas. De hecho utilizaban a los esclavos para cualquier tarea que no era digna de la razón.

\subsection{Sócrates}
Sócrates, Platón, Aristóteles son los tres grandes filósofos griegos. Cada uno tenía su visión sobre la matemática. Por ejemplo para Sócrates, ¿cómo se generaba conocimiento matemático? En uno de los diálogos, un esclavo va haciendo lo que Sócrates le dice. Le pide que trace un cuadrado, luego que busque la mitad de ese cuadrado. Al esclavo se le ocurre usar la mitad del lado, pero eso es un cuarto del área. Luego Sócrates lo va guiando hasta que llegan a la forma de llegar a esto, usando la mitad del lado y uniendo estos puntos. Esta sería la forma según Sócrates de un maestro: ir planteando un problema y a través de un diálogo va extrayendo de los estudiantes el conocimiento, es el método de la mayéutica. De ser partero de ideas. Según esta idea, el conocimiento está dentro nuestro y lo que hace el maestro es usar recursos para lograr sacar a la luz estas ideas.

\subsection{Platón}
El período clásico (s VII ~ XII) de la India, donde ya conviven los tres movimientos, es el que corresponde a los principales matemáticos de la India. Cabe destacar que en los Sulva Sutras ya nos encontrábamos con lo que conocemos como el teorema de Pitágoras (en un período muy anterior a Grecia). 

\subsection{Aristóteles}
El primero fue Aryabatha, que fue un matemático que propuso algoritmos para extraer raíces cuadradas y cúbicas. Trabaja con una aproximación del número $\pi$ de 3,1416. Es el creador de las primeras funciones trigonométricas. Las trabaja sobre un círculo, pero por ejemplo el seno no es cateto opuesto sino arco sobre hipotenusa. Esta función que propone se trabaja igual que el seno, tiene sus mismas características y las mismas propiedades. Hace tablas de esta función, similares a las actuales.

\section{Período Alejandrino (Helenístico)}
Este período comienza desde la invasión de los macedonios a Grecia dirigidos por Filipo II. Su hijo es Alejandro Magno. Alejandro, por influencia de Aristóteles tiene dos grandes sueños: tener un gran imperio (influencia también de su padre). Su imperio llega hasta el norte de la india. El segundo gran sueño de Alejandro era fundar una ciudad que fuera un gran polo cultural, un lugar donde se estudiaran ciencias y artes y donde fueran pensadores, científicos, artistas de distintos lugares del mundo. Esa ciudad fue Alejandría.

Luego de la muerte de Alejandro, su imperio se divide en tres partes entre tres generales: los que gobernaron la parte del norte de áfrica son los ptolomeos. Ellos llevan a Alejandría a su gran apogeo. Alejandría tuvo en su época la fama de ser el polo cultural que había soñado Alejandro. Muchos pensadores de esa época iban a Alejandría, estudiaban o vivían ahí, algunos volvían a sus ciudades, otros se quedaban otros iban y volvían. 

Había dos grandes edificios: uno era la biblioteca y el otro era el museo. La biblioteca llegó a tener más de 800 mil volúmenes. Tal magnitud en gran parte fue así ya que Alejandría al ser un puerto, el impuesto que se cobraba era la copia de los libros que traían los barcos. Llegaba el barco, dejaba el libro, los escribas lo copiaban y luego los devolvían. De esa forma, Alejandría logró la biblioteca que logró. 

Hasta ese momento se escribía en papiros, que eran muy frágiles. Aparece entonces el pergamino, que se hace con cuero de cordero. El cuero es sometido a un proceso químico donde se le quita la grasa y todo lo que se pueda llegar a pudrir y queda muy fino. En el pergamino se puede escribir y borrar incluso para volver a escribir. Entonces, colocaban esos cueros uno encima del otro, los doblaban, los cocían y empiezan a tener libros de la forma que conocemos, pero con hojas de pergamino.

La biblioteca de Alejandría fue quemada dos veces, una por los turcos y la otra por los romanos. En esas quemas se pierden muchos libros, pero algunos son rescatados y llevados a casas particulares, para ser llevados luego a Europa. Allí, van a los monasterios. En la Edad Media, muchos de los libros que llegaron de Alejandría fueron a parar a monasterios, abadías. Ahí, permanecen algunos enteros y otros no, dado que se podía borrar y hacía falta pergamino, que era muy caro. Le quitaban el piolín, lo limpiaban y lo reutilizaban para cosas que importaban en la Edad Media (textos religiosos, oraciones).

Lamentablemente, algunas de las obras que se salvaron de los incendios de la biblioteca, fueron perdidos en la Edad Media. Sin embargo, en los últimos años se descubrieron técnicas de rayos que permiten leer lo que estaba escrito debajo. Se comenzó esta tarea con pergaminos que estaban en museos. En el museo británico, con un pergamino medieval se dan cuenta que estaba escrito debajo y comienzan a leerlo con la técnica de rayos: era una obra de arquímedes. Eran unas pocas hojas ya que con la separación del original, no se conserva todo junto. 

Además de la biblioteca, existía el museo, que no es un museo de exhibición o exposición. Museo significa “donde habitan las musas”, las musas son las deidades inspiradoras de las ciencias y las artes. Había musas de la literatura, pintura, escultura y de las ciencias. El Museo Alejandrino era una especie de ciudad universitaria. 


\subsection{Euclides}
Entre la gente que transcurrió por el museo estaba Euclides. Él lo que hace es tomar la base e ideas de Aristóteles y llevarlas a la práctica. Escribió trece libros que son conocidos como “Los Elementos”. Lo que hace con estos libros es la organización de toda la matemática griega pero al estilo aristotélico, identificando nociones comunes, postulados, términos primitivos y empezando a deducir propiedades. El libro primero corresponde a geometría plana, cuyo esquema se mantiene hasta el día de hoy para los cursos de geometría (geometría euclidiana). Los elementos fueron libros de texto desde la época de Euclides (alrededor del 500 a.C.) hasta la actualidad en todas las universidades donde se estudia y estudiaba matemática. 

Debido a esta magnitud de divulgación, fue estudiado por mucha gente y le fueron encontrando muchas fallas, que se fueron subsanando a lo largo del tiempo. Una de ellas es que cuando Euclides empezó a trabajar, definió los términos primitivos. Por ejemplo el punto, la recta y el plano son definidos y son términos primitivos, cosa que para Aristóteles no se debía hacer para este tipo de entidades. En los elementos nos encontramos. Punto: es lo que no tiene partes. Recta: es la longitud sin anchura. Al principio se pensó que Euclides se había equivocado al hacer esto, pero en realidad no se trata de definiciones, sino de caracterizaciones. Actualmente la mayoría de los historiadores de la matemática reconocen que lo que hizo Euclides fue que lo hizo con fines didácticos, para que tratar de que el lector entendiera lo mismo que él (una especie de convención); quería que quedase sumamente claro lo que era cada elemento. Otra crítica que se le hace es en relación a los postulados. 

En el libro primero, geometría plana, Euclides declara cinco postulados. El quinto no tiene la característica que pedía Aristóteles de evidente. En la versión de Euclides, el quinto postulado dice: Dada una recta, cortada por dos transversales, si del mismo lado de la recta los ángulos que corresponden al mismo lado interior son menores que un recto cada uno, esas dos rectas se cortan de ese lado. Esto tiene más aspecto de teorema que de postulado; se ve necesaria su demostración (o al menos hacer una figura de análisis), es decir, no es evidente. Por ejemplo, el resto de los postulados si son evidentes: Todos los ángulos rectos son iguales entre sí, existen infinitos puntos. De hecho, Euclides no utiliza este postulado hasta la proposición 29, y siempre que puede no lo utiliza: prefiere utilizar una propiedad que utilizó ese postulado para demostrarse antes que el mismo postulado. 

Es por esta falta de evidencia, que muchos matemáticos intentaron demostrarlo. Y no lo lograron. A lo largo de la historia hay varios intentos de demostrarlo, hasta el siglo XIX. También hay intentos de reemplazarlo por otro. El matemático inglés Playfair (1748-1819) reformula el postulado tal cual lo conocemos hoy: Dada una recta y un punto que no está en ella, pasa una y solo una recta paralela a dicha recta. Ese quinto postulado dio mucho trabajo y controversia en la matemática, hasta que en siglo XIX, con su negación van a nacer las geometrías no euclidianas.

Algunas de las otras críticas que se le hacen a los Elementos de Euclides es tener algo mal definido, o alguna propiedad mal utilizada, o utilizar una propiedad que indirectamente está utilizando el quinto postulado, o usar sucesiones convergentes, que en la época de Grecia no existían.


\subsection{Los temores griegos. Infinitos}
Los dos temores nacen a partir de las paradojas de Zenón. Zenón es del período helénico, que lo que plantea son tres paradojas: la más conocida es la de Aquiles y la tortuga: deciden correr una carrera, él era el griego más veloz, entonces Aquiles le cede a la tortuga que empiece con una ventaja, por ejemplo de un metro. Entonces cuando Aquiles llega a donde estaba la tortuga, ésta ya había avanzado nuevamente. Cuando Aquiles llegue a este nuevo lugar, la tortuga va a haber avanzado otro poco y así sucesivamente. Quiere decir que nunca la alcanza, pero en la práctica Aquiles sería el gran ganador, entonces ¿dónde está la contradicción?. Después de mucho discutir dicen que está involucrado el infinito por tener que hacer infinitos pasos y también el movimiento. De aquí los dos temores griegos. 

Se dan cuenta que con su pensamiento racional no pueden manejar al movimiento y al infinito. Entonces, Aristóteles va a decir que tenemos 2 tipos de infinito: el potencial y el actual. El infinito potencial es la capacidad que tienen algunos objetos matemáticos de seguir creciendo. Por ejemplo si se piensa en los números naturales: si yo pienso en un número natural muy grande siempre puedo encontrar un número mayor. Es decir, puedo seguir haciéndolo crecer. Hoy en día se piensa en todos los naturales todos juntos. Ellos los pensaban de manera acotada sabiendo que tienen la posibilidad de ampliarlos. 

Otro ejemplo de este uso del infinito es como ve Euclides las rectas. En los Elementos, se encuentra que cuando se tiene una recta se piensa en una recta AB, y por ejemplo luego dice de prolongar la recta AB hasta el punto C. Luego hasta el punto D… en realidad lo que está haciendo es trabajar con segmentos que se pueden seguir prolongando. Estos ejemplos son del infinito potencial, la posibilidad de seguir creciendo. No se piensa en un infinito actual como se piensa hoy en día, hoy se toma la recta y decir que se prolonga la recta suena raro. Se puede prolongar un segmento dentro de la recta, pero la recta ya es un elemento infinito. 

Los griegos no se meten con este infinito actual, no piensan a la recta en su totalidad: los piensan con la característica de que pueden seguir creciendo.


\subsection{Diofanto}
Otro matemático de Alejandría es Diofanto. Él tiene una obra que se llama Aritmética, donde describe cómo resolver las ecuaciones que hoy conocemos como diofánticas: ecuaciones con coeficientes naturales o enteros y sus soluciones son números enteros.

\subsection{Apolonio}
Tiene una obra que se llama Las Cónicas, ahí describe cómo desde un doble cono se puede seccionar con planos y obtener lo que se conoce hoy en día como cónicas: elipse y circunferencia, hipérbola, parábola. Apolonio las estudia desde un punto de vista geométrico (no algebraico como hoy en día con matrices). En esa época se conciben aparatos con engranajes para construir cónicas.

\subsection{Arquímedes de Siracusa}
En Siracusa, gobernaba un tirano, Hierón que le encomendó a Arquímedes un problema, el de la corona que había mandado a hacer y desconfiaba de su composición, creía que el joyero lo había estafado y le había colocado más plata que oro de la solicitada. Entonces Arquímedes, solucionando esto, postula que todo cuerpo sumergido en un líquido recibe un empuje de abajo hacia arriba igual al peso del líquido desalojado. 

Además, calcula el volumen de una esfera. Para lograrlo, la corta en fetas o cintas de muy poco espesor, de manera que se pueda pensar a las fetas como un cilindro, entonces a cada cinta le calculó su volumen, como un cilindro y después las sumó todas. Esto es un antecesor de las integrales, que se comenzaron a utilizar en el siglo XVI. Arquímedes usa nociones parecidas a las de integrales, diferenciales, idea de infinito (porque esas secciones tenían que ser infinitas, es muy sugerente que no le tenía miedo al infinito). Cabe destacar que su pensamiento es muy distinto al pensamiento de los otros griegos. De hecho estas ideas no son valoradas y no van a ser utilizadas o retomadas por matemáticos de esa época.

Otra introducción al infinito que Arquímedes hace es tener una circunferencia, y dentro de esta inscribir (o circunscribir) polígonos: triángulo, cuadrado, pentágono… mientras más lados se tenga, más se va a acercar a la circunferencia. Entonces estos pequeños perímetros sumados se acercan al perímetro de la circunferencia. Llegan a la conclusión de que el perímetro es el diámetro por una constante, que todavía no se llama Pi (Euler la va a llamar así). Él propone encontrar esta constante con este método. Esto lo retoman posteriormente, por ejemplo en la cultura árabe.


\subsection{Los tres problemas clásicos}
En esa época descubren que una manera de construir figuras es con regla y compás. Entonces creen que todo se puede construir con regla y compás. Plantean que, dado un círculo, se quiere construir un cuadrado que tenga la misma área. Esto se llama la cuadratura del círculo. Otro problema que plantean es duplicar el volumen de un cubo: la duplicación del cubo. El tercer problema que plantean es la trisección del ángulo, de la misma manera que se forma la bisectriz. Son problemas que muchos siglos después se demuestran que no se pueden resolver: el primer problema involucra al número Pi que no es construible con regla y compás. El segundo, ecuaciones cúbicas que tampoco se resuelven con regla y compás. El tercero vuelve a involucrar números irracionales que no son construibles.

\section{Período Grecorromano}
El tercer período es el período grecorromano, de decadencia. Son invadidos por los romanos cuya visión sobre la matemática y la ciencia es totalmente distinta. Para los romanos solo vale todo aquello que tenga una aplicación concreta, todo aquello que sirve para construir algo. No son empíricos, más bien pragmáticos. 

Sus construcciones son monumentales y duraderas: acueductos, bóvedas, arcos. Para esto necesitan mucho cálculo. Lo que no sirve para este fin no es tenido en cuenta. Es decir, toda la búsqueda griega por construir cosas solo con regla y compás o la organización axiomática de la matemática no tiene valor. Lo poco que rescatan por ejemplo es: las proporciones del ser humano, son tenidas en cuenta en sus esculturas y artes. También resultados útiles para ellos como el teorema de Pitágoras. Les llamaba la atención y rescataron de este teorema los resultados con irracionales.

Vitruvio hace una recorrida y un listado de las cosas que se rescatan de Grecia: esto es enmarcado en un tratado de Arquitectura: proporciones, Pitágoras (el resultado de los dos catetos iguales a 1). Para Roma la mayor parte de la cultura griega no tiene importancia, por esto este período es considerado de decadencia.



\end{document}