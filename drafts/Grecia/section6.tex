Sócrates, Platón, Aristóteles son los tres grandes filósofos griegos. Cada uno tenía su visión sobre la matemática. Por ejemplo para Sócrates, ¿cómo se generaba conocimiento matemático? En uno de los diálogos, un esclavo va haciendo lo que Sócrates le dice. Le pide que trace un cuadrado, luego que busque la mitad de ese cuadrado. Al esclavo se le ocurre usar la mitad del lado, pero eso es un cuarto del área. Luego Sócrates lo va guiando hasta que llegan a la forma de llegar a esto, usando la mitad del lado y uniendo estos puntos. Esta sería la forma según Sócrates de un maestro: ir planteando un problema y a través de un diálogo va extrayendo de los estudiantes el conocimiento, es el método de la mayéutica. De ser partero de ideas. Según esta idea, el conocimiento está dentro nuestro y lo que hace el maestro es usar recursos para lograr sacar a la luz estas ideas.