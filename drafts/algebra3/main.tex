\documentclass[9pt,a4paper]{extarticle}
\usepackage[utf8]{inputenc}
\usepackage[spanish]{babel}
\usepackage[top=0.25in, bottom=0.25in, left=0.4in, right=0.4in]{geometry}
\usepackage{amsmath, amssymb, amsfonts}
\usepackage{graphicx}

%letra en arial
%\usepackage{helvet}
%\renewcommand{\familydefault}{\sfdefault}

\usepackage{enumerate}% http://ctan.org/pkg/enumerate
\newcommand{\plus}{\scalebox{0.6}{$+$}}

\begin{document}
\pagenumbering{gobble}
\hrule
\center{Álgebra III - Final A}
%\center {3/10/2017}
\begin{enumerate}
    \item 
    \begin{enumerate}
        \item Encontrar todos los naturales $n \in \mathbb{N}$ tales que $\dfrac{n+81}{2n-5}$ es un número natural.
        \item Determinar el mayor número natural $n$, sabiendo que el resto de dividir $n$ por 24 es el anterior de su cociente.
    \end{enumerate}
    
    \item 
    \begin{enumerate}
        \item Determinar $d=(-187,77)$ y hallar $r$ y $s$ tales que $d=-187r+77s$.
        \item ¿De cuántas maneras (especificando cuáles) puede descomponerse a 799 como suma de dos números positivos: uno múltiplo de 12 y otro múltiplo de 23?
    \end{enumerate}
    
    \item
    \begin{enumerate}
        \item Dar un ejemplo de dos enteros positivos diferentes de tal manera que cada uno de ellos tenga, exactamente, 600 divisores positivos. 
        \item Probar que todo primo de la forma $3k+1$ es de la forma $6q+1$.
        \item Resolver la ecuación $4x^2 \equiv 31(5)$
    \end{enumerate}
    
    \item
    \begin{enumerate}
        \item Probar que 13 divide al número $2^{70} + 3^{70}$.
        \item Hallar cuatro enteros consecutivos divisibles, respectivamente, por $5, 7, 9$ y $11$.
    \end{enumerate}
    
\end{enumerate}
\hrule

\center{Álgebra III - Final B}
%\center {3/10/2017}
\begin{enumerate}
    \item 
    \begin{enumerate}
        \item Demuestre que la unión de cualquier familia de funciones biyectivas de dominios y codominios respectivamente disjuntos dos a dos, es una función biyectiva que va de la unión de los dominios a la unión de los codominios.
        \item Enuncie un teorema en el que pueda utilizar la propiedad anterior.
    \end{enumerate}
    
    
    
        \item Demuestre que la unión de una familia finita y no vacía de conjuntos disjuntos dos a dos y coordinable con el $(0,1)$ es un conjunto coordinable con el $(0,1)$


    
    \item Demuestre que $\mathbb{R}$ es coordinable con el $(0,1)$
    
    \item Demuestre que $mcd(a,b)$ es una combinación lineal entera de $a$ y $b$
    
        \item Sean $a$ y $b$ dígitos. Demuestre que existe un múltiplo de $1557$ cuyo desarrollo decimal termina en $ab$.
    
    
\end{enumerate}
\hrule

\begin{enumerate}
    \item Demuestren los siguientes teoremas:
    \begin{enumerate}
        \item La diferencia entre un conjunto infinito y un conjunto finito, es un conjunto infinito.
        \item La unión de cualquier familia numerable de conjuntos finitos no vacíos, disjuntos dos a dos, es un conjunto numerable.
        \item La unión de cualquier familia finita y no vacía de conjuntos de potencia "c", disjuntos dos a dos, es un conjunto de potencia "c".
    \end{enumerate}
    
    \item Demuestren que el $mcd(a;b)$ es una combinación lin
\end{enumerate}
\end{document}