A Bhaskara se le atribuye la ecuación resolvente, o más específicamente, el reconocimiento de las dos raíces (la fórmula como procedimiento ya existía de antes). También se le atribuye el volver, o el utilizar las primeras demostraciones deductivas que están plasmadas en los Sulva Sutras; si bien utilizó demostraciones gráficas, fue uno de los primeros en utilizar demostraciones deductivas con lógica bivalente. También trabaja con series y sucesiones. De hecho, él no es el único: en la India hay mucho trabajo con sucesiones y series. 

El uso del infinito como un número los lleva a tener series que todas son convergentes, aunque converjan al infinito. Bhaskara opera muchas veces con el cero y con el infinito, y no le genera ningún tipo de problemas, ya que para la matemática de la India eran todos números operables. También reconoce que la raíz cuadrada de un número va a tener siempre dos resultados, positivo y negativo. Esto lo lleva a concebir que la cuadrática tenga dos soluciones.
