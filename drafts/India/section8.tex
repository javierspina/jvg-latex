El primero fue Aryabatha, que fue un matemático que propuso algoritmos para extraer raíces cuadradas y cúbicas. Trabaja con una aproximación del número $\pi$ de 3,1416. Es el creador de las primeras funciones trigonométricas. Las trabaja sobre un círculo, pero por ejemplo el seno no es cateto opuesto sino arco sobre hipotenusa. Esta función que propone se trabaja igual que el seno, tiene sus mismas características y las mismas propiedades. Hace tablas de esta función, similares a las actuales.