Brahmagupta trabajó conocimientos de astronomía en mayor medida y necesitó diseñar calendarios para medir el tiempo, hacer cuentas con números muy grandes. En determinado plantea una ecuación diofántica $61x^2 + 1 = y^2$ que tiene como soluciones más pequeñas a $x=226153980$ e $y=1766319049$. Usó, antes del siglo X aproximadamente, lo que hoy conocemos como ecuaciones de Pell, que fueron resueltas en occidente recién en el siglo XVII.