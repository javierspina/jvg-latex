Es tan o más conocido como poeta que como matemático. Logra enunciar que si tengo una ecuación de grado n, tiene a lo sumo n raíces, es un antecesor del teorema fundamental del álgebra sin considerar las raíces complejas. Hace aportes a la astronomía. Utiliza el triángulo de Pascal.

Tiene métodos geométricos para resolver ecuaciones cúbicas. Utiliza métodos para aproximar al número Pi, y logra hacerlo con bastantes decimales: no les interesa si es irracional o no porque no lo saben, si saben que tienen que seguir utilizando decimales, porque las aproximaciones que tienen no son suficientemente buenas. Estas aproximaciones las hacen con métodos de Arquímedes.
