Otro matemático árabe es Tabib ibn Quna, que es descendiente de sacerdotes babilónicos, y es uno de los principales traductores de los árabes, traduce todas las obras que encuentra de babilonio a árabe, pero también de otros idiomas, muchas desde el griego. Las traducciones no son solo de matemática sino que de medicina, botánica. Las traducciones que hace de matemática no son literales, sino que son comentadas, por ejemplo traduce los elementos de Euclides y va comentando si está de acuerdo o no con la demostración si le gusta, si lo hubiera hecho de otra forma, intenta probar el 5to postulado; y agregando aportes propios. 

Entre estos aportes, hay una generalización del teorema de Pitágoras: no solo construir cuadrados con los lados, sino cualquier figura, mientras sean semejantes, se mantiene el teorema. También hace una traducción de las cónicas de Apolonio, obras de Arquímedes (de hecho, muchos de los textos que llegan a nosotros llegan a través de estas traducciones árabes). También propone un método para medir longitudes de parábolas: dividía la parábola en pequeños segmentos y los rectificaba y de esa manera llega a medir la longitud de un arco de parábola. En occidente lo va a trabajar de forma similar Cavalieri.
