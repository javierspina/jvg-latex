Al llegar a la India, toman contacto con los sistemas de numeración y los llevan a todas sus zonas de influencia. Ellos adoptan este sistema de inmediato, le ven la ventaja por encima de otros sistemas de numeración. De hecho el sistema que utilizamos hoy en día se denomina arábigo por esto, aunque en realidad es indo-arábigo. Se dedicaron mucho al comercio, y en este ámbito muestran el sistema que utilizan y tratan de imponer pero no logran (en Europa -sur de España y sur de Italia- no lo logran).

Otro conocimiento en el que avanzan es en la geometría, sobre todo relacionado con la arquitectura. La arquitectura de los edificios árabes tiene un estilo determinado, muy geométrico, de mucha simetría. Trabajan mucho con mosaicos, en los mosaicos se encuentran diferentes movimientos geométricos (simetrías, rotaciones). También avanzan con respecto a la trigonometría, continúan el estudio de la India y definen las funciones trigonométricas de la manera que las entendemos hoy en día. Le encuentran muchas aplicaciones.

El nombre álgebra proviene del árabe. En una primera etapa la resolución de ecuaciones es geométrica, trabajan con figuras geométricas y después van generando algoritmos para trabajar en forma algebraica. En este comienzo el álgebra no hay un uso de variables, solamente se dedican a la operación, aunque después sí van apareciendo variables. 

Resuelven problemas aplicados al reparto de herencias, tienen leyes muy complejas al respecto de esto, se deben hacer muchos cálculos porque no es algo proporcional, había gente especializada en esto. 

En el área de trigonometría, definen las razones trigonométricas y van a descubrir propiedades, muchas de esas son identidades trigonométricas. El nombre seno de un ángulo proviene de una mala traducción, usaban jya, se tradujo como jaib que significa escote. Aplicaciones de la trigonometría para cálculos sobre triángulos rectángulos y oblicuángulos. Conocen el teorema del seno y del coseno.

Todos los conocimientos los usan en su vida diaria, en sus construcciones, y en el caso del sistema de numeración, cuando lo tratan de imponer, no lo logran, lo hacen del comercio haciendo cálculos mucho más rápido que alguien que usaba ábacos, en Europa no lo aceptan tan fácil. Es aceptado a partir de una obra de Fibonacci (1170-1240). 
