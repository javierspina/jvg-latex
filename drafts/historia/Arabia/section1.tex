Los árabes estuvieron ubicados en un territorio extenso, comienzan su época de viajes a partir del 622 d.C., que es el momento en el que Mahoma deja la ciudad de la Meca. En ese momento los árabes empiezan a viajar tratando de convertir a los distintos pueblos con los que iban teniendo contacto y a su vez, trataban de absorber la cultura que encontraban y traducirla a su lengua, ya que consideraban que todo debía ser traducido al árabe para después trabajar en esos conocimientos (científicos, matemáticos, lo que fueran). 

Se trata de un pueblo viajero, que lo hace desde el norte de la India hasta el sur de España: todo esto fue una zona de influencia árabe. Su forma de orientación era a través de la astronomía. Su astronomía es producto de tres culturas: Siria, Babilonia e India. Le dan mucho valor a la lectura, es obligación leer el Corán (es la única religión en donde el fundador no era analfabeta).
