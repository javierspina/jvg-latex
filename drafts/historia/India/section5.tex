La matemática que se maneja en actualidad en general ha recibido gran influencia de la matemática desarrollada en India. En India surgen dos elementos de la matemática que en occidente costaron mucho. Ahora, el infinito de la India es distinto. Hoy en día se habla de cardinales, de distintos tipos de infinito. En India hablaban de infinitos en direcciones: una, dos (el plano), en tres (espacio) o en infinitas direcciones. Todos estos infinitos de la India, si los terminamos analizando desde la teoría de Cantor, van a tener el mismo cardinal. En otro sentido, como el infinito es considerado un número, puedo operarlo como tal. Así empiezan a considerar por ejemplo que un número dividido infinito es igual a cero, sin el recurso del límite. Uno por cero es cero, uno por infinito, es infinito; dos dividido cero, es infinito. 

Si hay alguna indeterminación con respecto al resultado de una operación, pueden discutir quién tiene razón, y quien convenza mejor al otro, con mejor sustento teórico, será quien tenga razón (lógica polivalente).

El hecho de que el jainismo empiece a respetar estos dos principios hace que el vedismo no desaparezca, en realidad hay una convivencia de ambos movimientos. En la India, hasta hoy en día, existen muchas religiones y muchos dioses justamente basados en la idea del respeto.

Quienes llevan todos estos conocimientos a Europa serán los árabes. Pero no van a ser bienvenidos, porque llegan en plena Edad Media, donde escapan a la cultura pagana, entonces eso no puede estar bien. El conocimiento matemático de la India es mal recibido en un principio, incluso su sistema de numeración.
