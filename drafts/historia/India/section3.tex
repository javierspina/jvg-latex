El segundo movimiento es el Jainismo, que es un movimiento filosófico que no tiene origen religioso. Se basa en 2 principios. Uno de ellos es la pluralidad de ideas. Ellos aceptan que dos personas pueden pensar distinto y no por eso una de las dos está equivocada. No existe en el jainismo la lógica bivalente, sino que sus lógicas son polivalentes (en el vedismo usan lógicas bivalentes en sus demostraciones). Para ellos tiene razón el que convence mejor al otro: el que tiene mayor habilidad para persuadir es quien tiene argumentos más sólidos; no es una cuestión de contenido sino de cómo lo diga. El segundo principio es el respeto a esa pluralidad. Si considero que dos ideas distintas pueden ser correctas, entonces tengo que respetar ambas. En determinado momento trabajan con una lógica de cuatro valores, después llegan a siete valores. Su lógica al no ser bivalente hay algunos principios aristotélicos que no valen, por ejemplo el tercero excluido, el principio de no contradicción. Estos dos principios fundantes de la lógica occidental, no son válidos con lógicas polivalentes. 

El hecho de basarse en estos dos principios, hace que el pensamiento jainista cuando llega a una contradicción, no sea fuente de problemas. Por ejemplo el infinito, en el caso de Grecia, traía contradicciones y lo dejaron por fuera del análisis. Para los jainistas no es un problema pensar en el infinito. Entonces, comienzan a preguntarse cuántas almas existen: ya que creen en la reencarnación, dicen que simultáneo hay muchas almas en el universo, algunas reencarnadas en la Tierra, otras en otros planetas, otros lugares. Ante la pregunta de cuántas hay, llegan a responder que existen $2^{96}$ almas. 

Se hacen otro tipo de preguntas que involucran números muy grandes, por ejemplo si los dioses son muy rápidos cuando vuelan, ¿qué distancia pueden viajar en el tiempo de un parpadeo? Entienden que son distancias muy grandes y crean una unidad para eso que es lo que llaman la Shojana. Luego, se plantean un cubo que tenga una shojana de arista que se llena con vellones de corderos, y se preguntan cuánta es la cantidad de vellones. 

Este tipo de interrogantes los lleva a empezar a clasificar los números. Reconocen los números contables, los que se pueden llegar a contar; otra categoría son los números incontables que son tan grandes que incluso se pierde la noción de cuántos son. Otro tercer tipo de números son los infinitos, números aún más grandes que esos números incontables. Es decir, este reconocimiento de números reconoce al infinito como un número y también no tienen el temor de incluirlo en el pensamiento matemático. Siendo un número, se puede operar. 

La idea de infinito entonces, surge en India y la llaman Ananta. En la mitología de la India, Ananta es una serpiente que es guardiana de las puertas del infierno y solía ubicarse en un lago sagrado, y Visnú se sentaba arriba de ella para pasear por el lago. La serpiente se ponía en tal posición que termina desembocando en el símbolo del infinito. Si bien el que se usa hoy en día es el mismo, no tiene el mismo origen (que se debe a un matemático Inglés que no se sabe si tenía conocimiento de la matemática de la India).

Además del infinito, tienen el cero. Surge con el nombre de Sunya: en medio de todas las preguntas que se hacen, se empiezan a preguntar por el no-ser. Descubren que hay distintos tipos. Para ellos no es lo mismo no existir que no estar presente, no ser pensado, no haber sido concebido, etc. Identifican más de 20 tipos de no existencia. Una de las formas es la nada. Cuando identifican la nada necesitan simbolizarla de alguna manera y utilizan un pequeño círculo que recibe el nombre de Sunya. 

Este cero tiene tres funciones, una es la representación de la nada (es la primera cultura que necesita nominarla). Otra función es posicional: el sistema de numeración de la India es el heredado actualmente, tiene la misma función. Y la tercera es una función operatoria, que corresponde a ciertas operaciones que se hacen distintas si está presente el cero, como hoy en día. Por ejemplo la división por cero, cuando hay una potenciación a la cero que es igual a 1. El cero de la India es el primero en tener estas tres funciones, que hemos heredado.
