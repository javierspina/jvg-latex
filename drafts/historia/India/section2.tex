La cultura india tiene 3 movimientos filosóficos, el primero es el Vedismo, de origen religioso. El nombre proviene de Vedas que son los libros sagrados de la India. Los Vedas tienen en su parte principal al principio todo lo relacionado con lo mitológico y religioso y al final hay unos anexos, dos de los cuales cuales se relacionan con la matemática. Son los Kalpa Sutras y los Sulva Sutras. Los primeros tienen conocimientos de astronomía. La astronomía en la India es de origen lunar, todo es relacionado con la Luna. Estos conocimientos eran destinados a la astrología y también a la orientación. 

Los Sulva Sutras tienen conocimientos geométricos destinados a la construcción de altares, que tenían que tenían que tener ciertas dimensiones y formas, descritas en este anexo. Las propiedades son demostradas con maneras muy similares a las de Euclides: deductivas, con figuras de análisis, comparación de triángulos. Sacadas de contexto pueden parecer de Euclides, pero son anteriores. Su geometría entonces, queda atada a las construcciones que realizan. Su religión es politeísta, por eso tenían diferentes altares para distintos dioses, con diferentes características.

Aparte de ser una religión politeísta, creen en la reencarnación entonces socialmente están divididos en castas, cada una de ellas corresponde al nivel evolutivo que tienen. Algunas de las castas son sacerdotes, guerreros, artesanos, comerciantes y los parias (significa intocable, son seres inferiores, incluso inferiores a los animales). Ellos piensan que reencarnamos desde minerales, conciben una especie de rueda de reencarnaciones en la que se parte desde ser minerales (que para ellos tienen vida), vegetales, animales hasta seres humanos que viven en condiciones muy primitivas; todo esto en el primer cuadrante. El segundo cuadrante corresponde a la mayor parte de la población humana de la Tierra. El tercer cuadrante son también humanos pero que no forman familia con la idea de la mera procreación sino que que se forma familia con idea de amistad. El cuarto cuadrante son niveles evolutivos superiores que vienen a la Tierra para ayudar a los hombres, tienen potestad de decisión sobre venir o no a la Tierra, tienen millones de reencarnaciones y están evolucionados a menos que tengan algún karma negativo que los haga retroceder. 

Las demostraciones que se logran con los Sulva Sutras (muy parecidas al estilo Euclideano) se van perdiendo. Por alguna razón, dejan de demostrar y concurren a un período de demostraciones gráficas en las que no hay deducciones, hay simplemente figuras de análisis (se retoma la demostración deductiva en la época de Bhaskara).
