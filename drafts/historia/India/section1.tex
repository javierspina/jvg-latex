Los primeros pobladores de los que se tiene noción son los Harappa, que se asientan a lo largo del río Indo, alrededor del 3000 a.C.. No se tiene mucho conocimiento de que manejaban, pero si se sabe que se dedicaban a hacer ladrillos: eso genera ciertos conocimientos matemáticos, porque tienen que hacer mezclas y aparte los comerciaban. Posteriormente los Harappa van a ir desapareciendo y se van a ir fusionando con otras culturas que llegan y se va a formar una población que es la que da origen a la cultura india, de la antigüedad y de la actualidad (descendientes de estas fusiones). El idioma que manejan es el sánscrito.