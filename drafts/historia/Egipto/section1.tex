Parte de los conocimientos que transmitían lo hacían con jeroglíficos sobre papiros (fibras vegetales aplastadas para quitarles la humedad). Esa escritura durante mucho tiempo fue una lengua muerta hasta que luego de una expedición de Napoleón se encuentra la piedra roseta que tenía el mismo texto en jeroglífico y en griego (y otro idioma). A partir de este descubrimiento se empiezan a traducir muchos papiros. Algunos papiros tienen problemas matemáticos: uno de ellos es el papiro del Rhind, con varios problemas geométricos, planos y del espacio.

Los sacerdotes eran quienes decidían qué se enseñaba y que no. Parte de las enseñanzas se escribían en papiros y parte se transmitía en forma oral: los conocimientos que pertenecían a la clase sacerdotal y no querían que nadie más los sepa. Se transmitían oralmente y el método de aprendizaje era la repetición. Por lo tanto era una educación muy memorística, repetían hasta aprender. En algunos templos existían sacerdotisas, algunas mujeres entonces accedían a esos conocimientos, solo si iban a ser sacerdotisas de los templos. Cuando alguien no aprendía, lo azotaban porque pensaban que la espalda estaba conectada con los oídos.