Este período comienza desde la invasión de los macedonios a Grecia dirigidos por Filipo II. Su hijo es Alejandro Magno. Alejandro, por influencia de Aristóteles tiene dos grandes sueños: tener un gran imperio (influencia también de su padre). Su imperio llega hasta el norte de la india. El segundo gran sueño de Alejandro era fundar una ciudad que fuera un gran polo cultural, un lugar donde se estudiaran ciencias y artes y donde fueran pensadores, científicos, artistas de distintos lugares del mundo. Esa ciudad fue Alejandría.

Luego de la muerte de Alejandro, su imperio se divide en tres partes entre tres generales: los que gobernaron la parte del norte de áfrica son los ptolomeos. Ellos llevan a Alejandría a su gran apogeo. Alejandría tuvo en su época la fama de ser el polo cultural que había soñado Alejandro. Muchos pensadores de esa época iban a Alejandría, estudiaban o vivían ahí, algunos volvían a sus ciudades, otros se quedaban otros iban y volvían. 

Había dos grandes edificios: uno era la biblioteca y el otro era el museo. La biblioteca llegó a tener más de 800 mil volúmenes. Tal magnitud en gran parte fue así ya que Alejandría al ser un puerto, el impuesto que se cobraba era la copia de los libros que traían los barcos. Llegaba el barco, dejaba el libro, los escribas lo copiaban y luego los devolvían. De esa forma, Alejandría logró la biblioteca que logró. 

Hasta ese momento se escribía en papiros, que eran muy frágiles. Aparece entonces el pergamino, que se hace con cuero de cordero. El cuero es sometido a un proceso químico donde se le quita la grasa y todo lo que se pueda llegar a pudrir y queda muy fino. En el pergamino se puede escribir y borrar incluso para volver a escribir. Entonces, colocaban esos cueros uno encima del otro, los doblaban, los cocían y empiezan a tener libros de la forma que conocemos, pero con hojas de pergamino.

La biblioteca de Alejandría fue quemada dos veces, una por los turcos y la otra por los romanos. En esas quemas se pierden muchos libros, pero algunos son rescatados y llevados a casas particulares, para ser llevados luego a Europa. Allí, van a los monasterios. En la Edad Media, muchos de los libros que llegaron de Alejandría fueron a parar a monasterios, abadías. Ahí, permanecen algunos enteros y otros no, dado que se podía borrar y hacía falta pergamino, que era muy caro. Le quitaban el piolín, lo limpiaban y lo reutilizaban para cosas que importaban en la Edad Media (textos religiosos, oraciones).

Lamentablemente, algunas de las obras que se salvaron de los incendios de la biblioteca, fueron perdidos en la Edad Media. Sin embargo, en los últimos años se descubrieron técnicas de rayos que permiten leer lo que estaba escrito debajo. Se comenzó esta tarea con pergaminos que estaban en museos. En el museo británico, con un pergamino medieval se dan cuenta que estaba escrito debajo y comienzan a leerlo con la técnica de rayos: era una obra de arquímedes. Eran unas pocas hojas ya que con la separación del original, no se conserva todo junto. 

Además de la biblioteca, existía el museo, que no es un museo de exhibición o exposición. Museo significa “donde habitan las musas”, las musas son las deidades inspiradoras de las ciencias y las artes. Había musas de la literatura, pintura, escultura y de las ciencias. El Museo Alejandrino era una especie de ciudad universitaria. 
