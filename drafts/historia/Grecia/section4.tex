En el caso de Pitágoras se dice que no era una persona sino que era el nombre de una escuela (aunque mayormente se cree que era un individuo). Sucede que en esta escuela en particular, por el aprendizaje de Pitágoras en Egipto, suceden ciertos ritos de iniciación para esta escuela como los que hacían los sacerdotes en Egipto.

Las escuelas de pensamiento era una forma en la cual el conocimiento se transmitía en la sociedad griega. Se trataba de un pensador que llegaba a un espacio y empezaba a hablar, a este se le sumaban discípulos que lo escuchaban, acompañaban y discutían ciertos temas. Muchas veces esas escuelas de pensamiento generaban otras. Si había un desacuerdo, había una separación y si había un discípulo que sobresalía, se llevaba parte de la escuela con él. En cada una de las ciudades había varias escuelas. 

Para Pitágoras, el arjé es el número. Dice que todo en la naturaleza está regido por números y tienen personalidad, características humanas. Él trata de identificar las características de los números naturales. Por ejemplo dice que el 1 es el iniciador de todo, el origen, así como una persona puede ser un líder que inicie cosas. El 2 corresponde a la dualidad, el 3 a la perfección y así sucesivamente con distintos aspectos de personalidad. Esto sienta las bases de la numerología. Además de características de personalidad, hay características de familia, por ejemplo números primos, amigos, perfectos.

Pero para Pitágoras, estos no es la única clasificación. Existen números que representan figuras geométricas, los denominados números figurados: números que se pueden dibujar como figuras geométricas. Por ejemplo, números triangulares, que puedo dibujar con puntos siguiendo la forma de un triángulo. Esto les resultaba cómodo, porque escribían en la tierra con un palo. También hay cuadrados, y así sucesivamente con los polígonos regulares. 

Obtienen propiedades a partir de esto, de como formar un número en base al anterior. Por ejemplo, un triangular en base al anterior triangular. Ejemplo al 3, traje el uno y sume 2, al 6 traje 3 y sume 3, es decir el anterior más el índice. Se obtiene así una sucesión. O también otras propiedades, por ejemplo los cuadrados se pueden descomponer en números triangulares. 

Las demostraciones para esto son gráficas. La forma moderna sería con inducción completa. Pitágoras también observó que hay cosas en la naturaleza en las que se pueden hallar razones entre números. Ahí se encontró que aparecía muchas veces un número, el número de oro. Aparece en proporciones de crecimiento de hojas, del ser humano, de frutas, de flores. Dice que todo en la naturaleza está regido por ese tipo de razones. Por eso dice que el número áureo es el principio de todas las cosas, porque dentro de la naturaleza encuentra muchas razones entre números, números figurados y sobre todo el número áureo. 

Tiene una escala musical que la encuentra en base a medidas: tomando una medida y dividiendo sucesivamente por la mitad. La escala pitagórica es de 4 notas (Do Sol Mi Do). Fue usada luego en la edad media.

Pitágoras fue el primero en hacer una demostración para todos los triángulos rectángulos: el teorema de Pitágoras, el cuadrado de la hipotenusa es la suma de los cuadrados de los catetos. La demostración que se le atribuye es en la que se construyen los cuadrados sobre los lados. También trabaja con poliedros regulares, identifica inicialmente 4: cubo, octaedro, icosaedro, tetraedro. Luego se identifica el dodecaedro. Como inicialmente tenía 4 elementos, como los 4 elementos de la naturaleza, hace que se correspondan entre sí (cubo: tierra, octaedro: aire,  icosaedro: agua, tetraedro: fuego). El problema es cuando descubren el quinto (dodecaedro), que terminan por asignar al cosmos.

Irónicamente, la gran crisis de la escuela pitagórica surge con el teorema de Pitágoras. En una matemática y un universo donde todo es número, y el número entendido como razón entre números si se aplica el teorema de Pitágoras a un triángulo de catetos iguales a 1, se obtiene una hipotenusa raiz de 2. Ellos demuestran que no es un número que se pueda expresar como cociente de 2 números. Entonces, en un universo de números, aplicando un teorema que para ellos era muy importante, encuentran algo que no es un número (ya que para ellos el irracional no es numero) y les produce una crisis que se denominó la primera gran crisis de la matemática y a partir de eso, la escuela pitagórica empieza a desaparecer.