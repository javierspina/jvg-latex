Entre la gente que transcurrió por el museo estaba Euclides. Él lo que hace es tomar la base e ideas de Aristóteles y llevarlas a la práctica. Escribió trece libros que son conocidos como “Los Elementos”. Lo que hace con estos libros es la organización de toda la matemática griega pero al estilo aristotélico, identificando nociones comunes, postulados, términos primitivos y empezando a deducir propiedades. El libro primero corresponde a geometría plana, cuyo esquema se mantiene hasta el día de hoy para los cursos de geometría (geometría euclidiana). Los elementos fueron libros de texto desde la época de Euclides (alrededor del 500 a.C.) hasta la actualidad en todas las universidades donde se estudia y estudiaba matemática. 

Debido a esta magnitud de divulgación, fue estudiado por mucha gente y le fueron encontrando muchas fallas, que se fueron subsanando a lo largo del tiempo. Una de ellas es que cuando Euclides empezó a trabajar, definió los términos primitivos. Por ejemplo el punto, la recta y el plano son definidos y son términos primitivos, cosa que para Aristóteles no se debía hacer para este tipo de entidades. En los elementos nos encontramos. Punto: es lo que no tiene partes. Recta: es la longitud sin anchura. Al principio se pensó que Euclides se había equivocado al hacer esto, pero en realidad no se trata de definiciones, sino de caracterizaciones. Actualmente la mayoría de los historiadores de la matemática reconocen que lo que hizo Euclides fue que lo hizo con fines didácticos, para que tratar de que el lector entendiera lo mismo que él (una especie de convención); quería que quedase sumamente claro lo que era cada elemento. Otra crítica que se le hace es en relación a los postulados. 

En el libro primero, geometría plana, Euclides declara cinco postulados. El quinto no tiene la característica que pedía Aristóteles de evidente. En la versión de Euclides, el quinto postulado dice: Dada una recta, cortada por dos transversales, si del mismo lado de la recta los ángulos que corresponden al mismo lado interior son menores que un recto cada uno, esas dos rectas se cortan de ese lado. Esto tiene más aspecto de teorema que de postulado; se ve necesaria su demostración (o al menos hacer una figura de análisis), es decir, no es evidente. Por ejemplo, el resto de los postulados si son evidentes: Todos los ángulos rectos son iguales entre sí, existen infinitos puntos. De hecho, Euclides no utiliza este postulado hasta la proposición 29, y siempre que puede no lo utiliza: prefiere utilizar una propiedad que utilizó ese postulado para demostrarse antes que el mismo postulado. 

Es por esta falta de evidencia, que muchos matemáticos intentaron demostrarlo. Y no lo lograron. A lo largo de la historia hay varios intentos de demostrarlo, hasta el siglo XIX. También hay intentos de reemplazarlo por otro. El matemático inglés Playfair (1748-1819) reformula el postulado tal cual lo conocemos hoy: Dada una recta y un punto que no está en ella, pasa una y solo una recta paralela a dicha recta. Ese quinto postulado dio mucho trabajo y controversia en la matemática, hasta que en siglo XIX, con su negación van a nacer las geometrías no euclidianas.

Algunas de las otras críticas que se le hacen a los Elementos de Euclides es tener algo mal definido, o alguna propiedad mal utilizada, o utilizar una propiedad que indirectamente está utilizando el quinto postulado, o usar sucesiones convergentes, que en la época de Grecia no existían.
