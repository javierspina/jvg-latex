En esa época descubren que una manera de construir figuras es con regla y compás. Entonces creen que todo se puede construir con regla y compás. Plantean que, dado un círculo, se quiere construir un cuadrado que tenga la misma área. Esto se llama la cuadratura del círculo. Otro problema que plantean es duplicar el volumen de un cubo: la duplicación del cubo. El tercer problema que plantean es la trisección del ángulo, de la misma manera que se forma la bisectriz. Son problemas que muchos siglos después se demuestran que no se pueden resolver: el primer problema involucra al número Pi que no es construible con regla y compás. El segundo, ecuaciones cúbicas que tampoco se resuelven con regla y compás. El tercero vuelve a involucrar números irracionales que no son construibles.