Las culturas previas a Grecia eran empíricas. En la cultura griega hay un cambio de paradigma sobre qué es la ciencia y qué es la matemática, en ese cambio pasamos a dejar de tener lo que se llaman explicaciones míticas de los fenómenos y se pasa a tener explicaciones racionales. El griego busca explicar la realidad y todo lo que los rodea a través de la razón. Para los griegos, la razón es la que va a regir todo tipo de conocimiento y se va a constituir como algo característico del ser humano. 

A diferencia de otras culturas donde se buscaban resultados por ensayo y error, los griegos apuntan a generar una sistematización de conocimientos, una generalización de resultados. Buscan propiedades atribuibles a un conjunto en vez de ir por los casos particulares. Aparecen las demostraciones deductivas, una vez efectuada la demostración se va a saber para qué casos aplica.

Se distinguen tres períodos

\begin{enumerate}[topsep=0pt,itemsep=-1ex,partopsep=1ex,parsep=1ex]
    \item Clásico o Helénico: surgimiento del pensamiento racional (600 a.C. - 300 a.C.)
    \item Helenístico o Alejandrino: apogeo (300 a.C. - 300 d.C.)
    \item Grecorromano: decadencia (300 d.C. - 600 d.C.)
\end{enumerate}