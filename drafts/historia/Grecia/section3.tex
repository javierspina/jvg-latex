Para Thales, el arjé es el agua, dice que cualquier cosa del universo está compuesta de agua. A Thales se le atribuyen las primeras demostraciones racionales, esas demostraciones donde se enuncia un teorema y se trata de demostrarlo con el pensamiento deductivo. Sin embargo, en algunas de sus demostraciones incurren en lo que se denominan demostraciones circulares. Es decir, que para demostrar una propiedad B utiliza una propiedad A y para demostrar A vuelve a utilizar B.

Algo que le dio mucho prestigio a Thales fue haber predecido el primer eclipse. En su búsqueda de tratar de explicar fenómenos, empezó a tener un modelo del movimiento del Sol, la Tierra y la Luna. Luego logró explicar un eclipse de Sol, diciendo que la Luna se ubica entre la Tierra y el Sol, además de poder predecirlo para el 28 de mayo del 585 a.C.. Esta predicción fue afamada y le dió prestigio porque el eclipse se dió durante una batalla contra los macedonios: el enemigo quedó confundido por el fenómeno, pero los griegos al saber del rumor, pudieron no distraerse tanto y así derrotaron a los macedonios fácilmente. La credibilidad que ganó Thales permitió que su modelo astronómico sea respetado. Fue incluido en varios listados de siete sabios griegos (de hecho, fue el único presente en todos los listados).

Otro modelo que tiene Thales es de los imanes. Él explica que dentro de un imán hay unos pequeños imanes. Por ejemplo con el fenómeno de la imantación del hierro: dentro de este material los pequeños imanes están desordenados, pero al estar en contacto con un imán, esos imanes del hierro se van ordenando, obteniendo así, un imán artificial. Es un modelo similar al actual. Decía que los pequeños imanes están a nivel atómico. De hecho, la palabra átomo viene de esta época, de Demócrito. Demócrito explica a la materia, dice que cualquier elemento que tenga lo puedo dividir sucesivamente hasta llegar a un punto que no pueda seguir dividiendo porque llego a algo elemental que no tiene partes. Átomo es eso, aquello que no tiene partes, la menor división posible de la materia. En el modelo de Thales está esta idea de átomo, los pequeños imanes están en este nivel.

Otro aporte de Thales es el Teorema de las rectas paralelas. Una aplicación que hace es para medir la altura de la pirámide de Keops. Para esa época resultaba muy difícil medir una pirámide. Thales visita Egipto, donde le plantean el problema de medir la pirámide. Él clava un bastón y observa las sombras del bastón y la pirámide. Con los datos de las sombras y mediante un esquema, plantea una proporcionalidad de los lados.