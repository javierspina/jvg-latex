Tiene una obra que se llama Las Cónicas, ahí describe cómo desde un doble cono se puede seccionar con planos y obtener lo que se conoce hoy en día como cónicas: elipse y circunferencia, hipérbola, parábola. Apolonio las estudia desde un punto de vista geométrico (no algebraico como hoy en día con matrices). En esa época se conciben aparatos con engranajes para construir cónicas.