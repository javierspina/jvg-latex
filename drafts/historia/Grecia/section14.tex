En Siracusa, gobernaba un tirano, Hierón que le encomendó a Arquímedes un problema, el de la corona que había mandado a hacer y desconfiaba de su composición, creía que el joyero lo había estafado y le había colocado más plata que oro de la solicitada. Entonces Arquímedes, solucionando esto, postula que todo cuerpo sumergido en un líquido recibe un empuje de abajo hacia arriba igual al peso del líquido desalojado. 

Además, calcula el volumen de una esfera. Para lograrlo, la corta en fetas o cintas de muy poco espesor, de manera que se pueda pensar a las fetas como un cilindro, entonces a cada cinta le calculó su volumen, como un cilindro y después las sumó todas. Esto es un antecesor de las integrales, que se comenzaron a utilizar en el siglo XVI. Arquímedes usa nociones parecidas a las de integrales, diferenciales, idea de infinito (porque esas secciones tenían que ser infinitas, es muy sugerente que no le tenía miedo al infinito). Cabe destacar que su pensamiento es muy distinto al pensamiento de los otros griegos. De hecho estas ideas no son valoradas y no van a ser utilizadas o retomadas por matemáticos de esa época.

Otra introducción al infinito que Arquímedes hace es tener una circunferencia, y dentro de esta inscribir (o circunscribir) polígonos: triángulo, cuadrado, pentágono… mientras más lados se tenga, más se va a acercar a la circunferencia. Entonces estos pequeños perímetros sumados se acercan al perímetro de la circunferencia. Llegan a la conclusión de que el perímetro es el diámetro por una constante, que todavía no se llama Pi (Euler la va a llamar así). Él propone encontrar esta constante con este método. Esto lo retoman posteriormente, por ejemplo en la cultura árabe.
