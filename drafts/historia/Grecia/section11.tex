Los dos temores nacen a partir de las paradojas de Zenón. Zenón es del período helénico, que lo que plantea son tres paradojas: la más conocida es la de Aquiles y la tortuga: deciden correr una carrera, él era el griego más veloz, entonces Aquiles le cede a la tortuga que empiece con una ventaja, por ejemplo de un metro. Entonces cuando Aquiles llega a donde estaba la tortuga, ésta ya había avanzado nuevamente. Cuando Aquiles llegue a este nuevo lugar, la tortuga va a haber avanzado otro poco y así sucesivamente. Quiere decir que nunca la alcanza, pero en la práctica Aquiles sería el gran ganador, entonces ¿dónde está la contradicción?. Después de mucho discutir dicen que está involucrado el infinito por tener que hacer infinitos pasos y también el movimiento. De aquí los dos temores griegos. 

Se dan cuenta que con su pensamiento racional no pueden manejar al movimiento y al infinito. Entonces, Aristóteles va a decir que tenemos 2 tipos de infinito: el potencial y el actual. El infinito potencial es la capacidad que tienen algunos objetos matemáticos de seguir creciendo. Por ejemplo si se piensa en los números naturales: si yo pienso en un número natural muy grande siempre puedo encontrar un número mayor. Es decir, puedo seguir haciéndolo crecer. Hoy en día se piensa en todos los naturales todos juntos. Ellos los pensaban de manera acotada sabiendo que tienen la posibilidad de ampliarlos. 

Otro ejemplo de este uso del infinito es como ve Euclides las rectas. En los Elementos, se encuentra que cuando se tiene una recta se piensa en una recta AB, y por ejemplo luego dice de prolongar la recta AB hasta el punto C. Luego hasta el punto D… en realidad lo que está haciendo es trabajar con segmentos que se pueden seguir prolongando. Estos ejemplos son del infinito potencial, la posibilidad de seguir creciendo. No se piensa en un infinito actual como se piensa hoy en día, hoy se toma la recta y decir que se prolonga la recta suena raro. Se puede prolongar un segmento dentro de la recta, pero la recta ya es un elemento infinito. 

Los griegos no se meten con este infinito actual, no piensan a la recta en su totalidad: los piensan con la característica de que pueden seguir creciendo.
