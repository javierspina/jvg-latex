En el período helénico se sientan las bases del pensamiento griego. Surgen los llamados pensadores presocráticos, que buscan el principio de todas las cosas, en el intento de la explicación racional, el llamado arjé. A partir de esta búsqueda es que dan explicaciones de los distintos fenómenos que encuentran. Thales de Mileto y Pitágoras de Samos son de este período.