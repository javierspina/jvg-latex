En el caso de Platón, los objetos matemáticos tienen existencia en el mundo de las ideas. Lo que vemos alrededor nuestro son reflejos de estos objetos. No se tiene un rectángulo, se tiene una hoja con forma rectangular, que refleja lo que el rectángulo es en el mundo de las ideas. También son parte de su obra la alegoría de la caverna, que tiene que ver como nosotros accedemos el mundo de las ideas: meditando, filosofando, preguntandonos cosas. Según Platón, lo más cercano a las ideas es la matemática porque es la disciplina o ciencia que trabaja con objetos abstractos, por lo tanto está trabajando con el mundo de las ideas.