El tercer período es el período grecorromano, de decadencia. Son invadidos por los romanos cuya visión sobre la matemática y la ciencia es totalmente distinta. Para los romanos solo vale todo aquello que tenga una aplicación concreta, todo aquello que sirve para construir algo. No son empíricos, más bien pragmáticos. 

Sus construcciones son monumentales y duraderas: acueductos, bóvedas, arcos. Para esto necesitan mucho cálculo. Lo que no sirve para este fin no es tenido en cuenta. Es decir, toda la búsqueda griega por construir cosas solo con regla y compás o la organización axiomática de la matemática no tiene valor. Lo poco que rescatan por ejemplo es: las proporciones del ser humano, son tenidas en cuenta en sus esculturas y artes. También resultados útiles para ellos como el teorema de Pitágoras. Les llamaba la atención y rescataron de este teorema los resultados con irracionales.

Vitruvio hace una recorrida y un listado de las cosas que se rescatan de Grecia: esto es enmarcado en un tratado de Arquitectura: proporciones, Pitágoras (el resultado de los dos catetos iguales a 1). Para Roma la mayor parte de la cultura griega no tiene importancia, por esto este período es considerado de decadencia.
