\begin{comment}
-------------------------
----------Plot-----------
-------------------------
Insert section plot from 1-plot.text here
-Section 1-
Max X words

----AIM----




--Details--




-------------------------
------Senses Check-------
-------------------------
Smell
Touch
Sound
Taste
Sight

-------------------------
------Other Checks-------
-------------------------
Checked adverb use? (0)
Checked cliche use? (0)
Checked tense integrity?
Checked perspective integrity?
Checked reuse of major words?
Checked sentence length?
Checked simile use? (<=5)
Checked metaphor use? (<=3)
Checked description length?
Checked paragraph density

Zombie Ipsum text from here: http://www.zombieipsum.com/
\end{comment}


La cultura china es muy antigua, pero tuvo muy poca influencia sobre la matemática que se maneja hoy en día en occidente. China se mantuvo aislada, con la Gran Muralla, y todos los conocimientos que fueron desarrollando nunca pasaron sus fronteras y así no llegaron a occidente. Se trata de una cultura de origen fluvial, se ubican a lo largo de Hoangho Y el Tsequian (río Rojo y río Verde). Allí se ubican los primeros pobladores de China de los que se tenga conocimiento.

Las primeras poblaciones son aproximadamente del 2000 a.C. y se van a ir formando especies de ciudades feudales, muy similar a la organización de la edad media de occidente. Si bien existía un emperador (dinastía Shang), no se logró unificar toda la China y se mantuvieron bajo su dominio pero al estilo feudo. El primer emperador que comienza con la unificación es Qiu, que es el mismo que construye la Gran Muralla. La idea de la muralla era separar los lugares que se iban conquistando del resto del mundo. Lo que más le preocupaba eran los mongoles, que habían quedado del otro lado y continuamente trataban de invadirlos; la muralla es construida en parte por protección y también para delimitar aquello que era su imperio. Qiu fue un déspota que a quienes no puede conquistar los asedia y destruye.

Socialmente, el orden de importancia social por debajo del emperador eran los mandarines -clase alta- que son los empleados estatales. Existió una burocracia muy grande en la antigua China. Llegar a ser mandarín no era por herencia, sino que existía un exámen que se tenía que dar para llegar a esos puestos. Eso generó que haya escuelas para preparar a los mandarines para dar ese exámen. Aprendían a llevar la contabilidad del Estado, historia, literatura, a manejar los sistemas de numeración. Aparte como buenos burócratas, tienen que atender al público, entonces también aprenden lo que es diplomacia y relaciones públicas. Para asistir a escuelas de mandarines no necesitaban pertenecer a esa clase, pero si eran escuelas pagas, por lo que se necesitaba dinero.
Inventaron la brújula, la pólvora, papel, tinta pero China permanece aislada hasta el siglo XV. Marco Polo pone en contacto a China con Occidente, pero el contacto es especial, ya que desde el Occidente ven a China como exótica y en cuanto a conocimiento científico es un choque cultural muy grande porque su filosofía es distinta, de esa manera la ciencia de China no va a ser aprovechada. 
