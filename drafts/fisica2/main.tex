\documentclass[a4paper, 12pt]{article}
\usepackage{geometry}
 \geometry{
 a4paper,
 total={170mm,257mm},
 left=20mm,
 top=20mm,
 }
\usepackage[utf8]{inputenc}
\usepackage[spanish]{babel}
\usepackage{amsmath, amssymb, amsfonts}

\newcommand\setItemnumber[1]{\setcounter{enumi}{\numexpr#1-1\relax}}

\title{Ejercicios Física 2°C}


\begin{document}

\section*{Ejercicios del capítulo 2: Movimiento en Línea Recta}
\subsection*{Desplazamiento, tiempo y velocidad media}
\begin{enumerate}
\setItemnumber{1}
  \item Un cohete que lleva un satélite acelera verticalmente alejándose de la superficie terrestre. 1,15 s después del despegue, el cohete libra el tope de su plataforma, 63 m sobre el suelo; después de otros 4,75 s, está 1,00 km sobre el suelo. Calcule la magnitud de la velocidad media del cohete en a) la parte de 4,75 s de su vuelo; b) los primeros 5,90 s de su vuelo.
\setItemnumber{7}
  \item a) Su vieja combi traquetea con una rapidez media de 8,0 m/s durante 60 s, luego entra en calor y corre otros 60 s con una rapidez media de 20,0 m/s. a) Calcule la rapidez media en los 120 s. b) Suponga que la rapidez de 8,0 m/s se mantuvo durante 240 m, seguida de la rapidez media de 20,0 m/s durante otros 240 m. Calcule la rapidez media en toda la distancia. c) ¿En cuál caso es la rapidez media de todo el movimiento el promedio de las dos rapideces?
\end{enumerate}
\subsection*{Velocidad Instantánea}
\begin{enumerate}
  \setItemnumber{9}
  \item Un auto está parado ante un semáforo. Después viaja en línea recta y su distancia respecto al semáforo está dada por $ x(t) = bt^{2} - ct^{3} $, donde $ b = 2,40$  $m/s^{2} $ y $ c = 0,120$  $m/s^{3} $. a) Calcule la velocidad media del auto entre $ t = 0$ y $ t = 10,0 s $. b) Calcule la velocidad instantánea en i)$ t = 0$; ii)$ t = 5,0 s$; iii)$ t = 10,0s$. c) ¿Cuánto tiempo después de arrancar vuelve a estar parado el auto?
    \end{enumerate}
\subsection*{Aceleración media e instantánea}
\begin{enumerate}
  \setItemnumber{11}
  \item 
    
  \setItemnumber{29}
  \item
    
  \setItemnumber{33}
  \item
    
  \setItemnumber{35}
  \item
    
  \setItemnumber{43}
  \item
    
  \setItemnumber{55}
  \item
  
  \setItemnumber{65}
  \item
  
  \setItemnumber{69}
  \item
  
  \setItemnumber{75}
  \item
    
  \setItemnumber{77}
  \item
  
  \setItemnumber{89}
  \item
\end{enumerate}

\end{document}
